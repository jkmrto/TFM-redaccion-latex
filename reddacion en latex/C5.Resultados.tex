%%%%%%%%%%%%%%%%%%%%%%%%%%%%%%%%%%%%%%%%%%%%%%%%%%%%%%%%%%%%%%%%%%%
%%% Documento LaTeX 																						%%%
%%%%%%%%%%%%%%%%%%%%%%%%%%%%%%%%%%%%%%%%%%%%%%%%%%%%%%%%%%%%%%%%%%%
% T�tulo:		Cap�tulo 2
% Autor:  	Ignacio Moreno Doblas
% Fecha:  	2014-02-01
% Versi�n:	0.5.0
%%%%%%%%%%%%%%%%%%%%%%%%%%%%%%%%%%%%%%%%%%%%%%%%%%%%%%%%%%%%%%%%%%%
\chapterbegin{Resultados}
\label{chp:Resultados}


\section{Clasificicaci�n}

\subsubsection{M�tricas de Evaluaci�n}

\par Hay diferentes medidas de precisi�n  diagn�stica que se relacinan con los diferentes aspectos del procedimiento del diagn�stico. Algunas se utilizan para evaluar la capacidad discriminativa de la prueba y otras para estimar su propiedad de predicci�n.


\begin{table}[]
\centering
\label{my-label}
\begin{tabular}{C{3cm}C{3cm}|C{3cm}|C{3cm}|}
\cline{3-4}
  &  & \multicolumn{2}{c|}{	\textbf{Condici�n Real}} \\ 
\cline{3-4} 
  &  & \textbf{Positivo} & \textbf{Negativo} \\ 
\hline
\multicolumn{1}{|c|}{\multirow{2}{*}{\textbf{Predicci�n}}} & \textbf{Positivo} &  Verdadero Positivo (VP)                 &               Falso    Positivo (FP) \\ \cline{2-4} 
\multicolumn{1}{|c|}{}                            & \textbf{Negativo}     &  Falso Negativo (FN)               & Verdadero Negativo (VN)              \\ \hline
\end{tabular}
\caption{Tabla de Nomenclatura Estad�stica en Clasificaci�n}
\end{table}

\subsubsection{f1\_score}
\subsubsection{recall score}
\subsubsection{accuracy}
\subsubsection{area under the curve}
\subsubsection{precision}
\subsubsection{?}

%\minitoc
\chapterend{}