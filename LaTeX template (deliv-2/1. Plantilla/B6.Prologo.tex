%%%%%%%%%%%%%%%%%%%%%%%%%%%%%%%%%%%%%%%%%%%%%%%%%%%%%%%%%%%%%%%%%%%
%%% Documento LaTeX 																						%%%
%%%%%%%%%%%%%%%%%%%%%%%%%%%%%%%%%%%%%%%%%%%%%%%%%%%%%%%%%%%%%%%%%%%
% T�tulo:	Pr�logo
% Autor:  Ignacio Moreno Doblas
% Fecha:  2014-02-01
%%%%%%%%%%%%%%%%%%%%%%%%%%%%%%%%%%%%%%%%%%%%%%%%%%%%%%%%%%%%%%%%%%%

\chapterbeginx{Pr�logo}

%\minitoc
	Aqu� debe escribirse el pr�logo del proyecto fin de carrera.
	\medskip
	
	La calidad en la presentaci�n de los textos y las flexibilidad de \LaTeX\ me llevaron a aprenderlo, a pesar de su dif�cil curva de aprendizaje.\nli
	Espero que esta plantilla ayude notablemente a suavizar este inconveniente.
	
	Quiero agradecer a las personas que han colaborado en la realizaci�n de esta plantilla \LaTeX. Es un sistema muy r�pido y c�modo en la generaci�n de este tipo de documentos t�cnicos y su lectura es francamente agradable.
	
	Animo a todo el mundo a utilizarlo.
	
	Desde la p�gina de la escuela hay disponible tambi�n un \miindex{manual de estilo} para ayudar en la redacci�n y el acabado del proyecto.
	Puede consultarse en~\cite{GuiaEstilo}~\footnote{
		\url{http://www.uma.es/media/files/Manual_de_Estilo_TFG_ETSIT.pdf}
	}.

	Tambi�n ser�a interesante hacer dos manuales m�s: 
\begin{itemize}
	\item{Uno de \LaTeX, que explique con m�s detalle c�mo utilizar este sistema. Aunque en Internet hay muchos disponibles, un manual r�pido y directo suavizar�a a�n m�s la curva de aprendizaje.\nli
		Quiz�, lo m�s importante es que integre todos los elementos que un usuario necesita, ya que normalmente es necesario acudir a varias fuentes y eso suele requerir demasiado tiempo.\nli
		El cap�tulo~\ref{chp:ManLaTeX} contiene informaci�n orientada a un iniciado en este sistema.}
	
	\item{Y otro, que explique herramientas y m�todos �tiles que un proyectando puede necesitar en la elaboraci�n del proyecto, tal como llevar un control de versiones de la documentaci�n o el c�digo fuente desarrollado utilizando Assembla\TM. Este �ltimo manual es interesante tambi�n para muchos j�venes profesionales, especialmente en el �rea de desarrollo de sistemas.}
\end{itemize}

	Me reservo el derecho de hacerlo, dado el escaso tiempo del que dispongo.
	\bigskip
	
	Muchas gracias.

\chapterend