
\begin{thebibliography}{9}

\bibitem{Alzheimer1} 
	Henley, David B.,
	Sundell, Karen L.,
	Sethuraman, Gopalan,
 	Siemers, Eric R.,
	2011,	
	\textit{Safety profile of Alzheimer's disease populations in Alzheimer's Disease Neuroimaging 			Initiative and other 18-month studies},
 	407-416

\bibitem{ReservaCognitiva}
	Rodríguez Álvarez M., Sánchez J. L.,
	2004,
	\textit{Reserva cognitiva y demencia},
	2004, vol. 20, nº 2 (diciembre), 175-186 
	
\bibitem{DiagnosticoNormal}
 Petrella J.R, Coleman R.E., Doraiswamy P.M., Neuroimaging and Early Diagnosis
of Alzheimer Disease: A Look to the Future. Radiology 2003; 226:315?336.



\bibitem{MRIsurvey}
	Rémi Cuingnet et al, 
	\textit{Automatic classification of patients with Alzheimer's disease from
	structural MRI: A comparison of ten methods using the ADNI database},
	 Neuroimage,
	vol. 56, no. 2, pp. 766-781, 2011.

\bibitem{introPET}
	W. Cai, 
	D. Feng, 
	R. Fulton, 
	\textit{Content-based retrieval of dynamic PET functional images} 
	IEEE Trans. Inf. Technol. Biomed., vol. 4, no. 2, pp. 152-158, 2000.

\bibitem{PETMRIMultimodal}
	Ortiz, A.; 
	Fajardo, D.; 
	Górriz, J.M.; 
	Ramírez, J.; 
	Martínez-Murcia, F.J.,
	\textit{Multimodal
image data fusion for Alzheimer's Disease diagnosis by sparse representation}, 
International Conference on Innnovation in Medicine and Healthcare (InMed), 2014


\bibitem{PETMRIMultimodal2}
	Daoqiang Zhanga, 
	Yaping Wanga, 
	Luping Zhoua, 
	Hong Yuana, 
	Dinggang Shena,
	\textit{Multimodal Classification of Alzheimer's Disease and Mild Cognitive Impairment}
	Neuroimage. 2011 April 1; 55(3): 856-867


\bibitem{SVMtrees}
	D. Salas-Gonzalez, 
	J. M. Gorriz, 
	J. Ramírez, 
	M. Lopez, 
	I Alvarez,
	\textit{Compute-aided diagnosis of Alzheimer's disease using support vector machines and classification trees}
	Phys. Med. Biol. 55 (2010) 2807-2817


\bibitem{Survey}
	Ruaa Adeeb Abdulmunem Al-falluji
	\textit{MRI based Techniques for Detection of Alzheimer: A Survey}
	International Journal of Computer Applications (0975 - 8887)
	Volume 159 - No 5, February 2017

\bibitem{Survey2}
	S.Mareeswari1, 
	Dr.G.Wiselin,
	\textit{A survey Early Detection of Alzheimer's Disease using different techniques}
	International Journal on Computational Science and Applications (IJCSA) Vol.5, No.1,February 2015

\bibitem{CoD}
Bellman RE, 
1961,
\textit{Adaptive control processes: a guided tour}, Princeton
University Press.


\bibitem{CoD2}
	David L. Donoho
	Department of Statistics
	\textit{High-Dimensional Data Analysis: The Curses and Blessings of Dimensionality}
	August 8, 2000

\bibitem{ReduceDimensionality}
	Benson Mwangi, 
	Tian Siva Tian,  
	Jair C. Soares
	\textit{A review of feature reduction techniques in neuroimaging}
	Neuroinformatics. 2014 April ; 12(2): 229-244. doi:10.1007/s12021-013-9204-3.

\bibitem{DeepLearning1}
	Siqi Liu, 
	Sidong Liu,
	\textit{EARLY DIAGNOSIS OF ALZHEIMER'S DISEASE WITH DEEP LEARNING}

\bibitem{DeepLearning2}
	Saman S., 
	Ghassem T.,
	\textit{Classification of Alzheimer's Disease Structural MRI Data by Deep Learning Convolutional Neural Networks}
	22 Jul 2016

\bibitem{AutoEncoderVariational}
	Diederik P. Kingma,
	Max Welling,
	\textit{Auto-Encoding Variational Bayes}
	1 May 2014

\bibitem{AutoEncoderOrigin1}
	 D. E. Rumelhart, G. E. Hinto, and R. J. Williams. 
	 Learning Internal Representations by Error Propagation
	 9 October 1986

\bibitem{AutoEncoderSurvey}
	: I. Guyon, 
	G. Dror, 
	V. Lemaire, 
	G. Taylor and 
	D. Silver
	Autoencoders, Unsupervised Learning, and Deep Architectures
	2012

\bibitem{AutoEncoderDenoising}
	P. Vincent
	H. Larochelle
	I. Lajoie
	Stacked Denoising Autoencoders: Learning Useful Representations in a Deep Network with a Local Denoising Criterion
	2010

\bibitem{VAE_gen}
ADVERSARIAL EXAMPLES FOR GENERATIVE MODELS

\bibitem{DeepGenerativeModels}
	Danilo J. Rezende, 
	Shakir Mohamed, 
	Daan Wierstra
	Stochastic Backpropagation and Approximate Inference in Deep Generative Models

\bibitem{AD_historic}
	Bennett, D. A.,
	Evans, D. A., 
	1922.
	Alzheimer's Disease.
	Disease-a-Month 38(1), 7-64.

\bibitem{ADhistoric2}
	Nakako, S.,
	Kato, T.,
	Nakamura.,
	1996.
	Acetylcholinesterase activity in cerebrospinal fluid of patients with alzhimer's disease and senile dementia.
	Journal of the Neurological Sciences
	75(2).
\bibitem{mitocondria2}
	Rodriguez-Violante M.,
	Cervantes A.,
	Vargas S.,
	2010.
	Papel de la función mitocondrial en las enfermedades neurodegenerativas.
	Arch Neurocien (Mex)  Vol. 15, Nº1: 39-46

\bibitem{mitocondriaGeneral}
	Swerdlow, R.,
	2011
	Brain agin, alzheimer's diseas, and mitochondira.
	Biochim Biophys Acta 1812(12), 1630-1639


\bibitem{Nations}
	Nations U.,
	2008
	Department of economic and social affairs, world population prospects. 

\bibitem{Predemencia1}
	 Arnáiz E,., 
	 Almkvist O., 
	 2003,. 
	 Neuropsychological features of mild cognitive impairment and preclinical Alzheimer's disease. 



\bibitem{Predemencia2}
 	Palmer K., 
 	Berger A. K., 
 	Monastero R., 
 	Winblad B., 
 	Bäckman L., 
 	Fratiglioni L. 
 	2007. 
 	Predictors of progression from mild cognitive impairment to Alzheimer disease. 
 	Neurology 68 (19): 1596-1602. 
 	PMID 17485646. doi:10.1212/01.wnl.0000260968.92345.3f.

\bibitem{DemenciaMemoria}
	 Carlesimo GA, Oscar-Berman M (junio de 1992). «Memory deficits in Alzheimer's patients: a comprehensive review». Neuropsychol Rev 3 (2): 119-69. PMID 1300219.

 \bibitem{DemenciaComunicacion}
 	 Frank EM (septiembre de 1994). «Effect of Alzheimer's disease on communication function». J S C Med Assoc 90 (9): 417-23. PMID 7967534.
\bibitem{ADconducta}
	Volicer L, 
	Harper DG, 
	Manning BC, 
	Goldstein R, 
	Satlin A., 
	Mayo de 2001. 
	Sundowning and circadian rhythms in Alzheimer's disease». Am J Psychiatry 158 (5): 704-711. PMID 11329390. 


\bibitem{hipocampo}
	Mu Y., 
	Gage FH,
	2011 Dec,
Mol Neurodegener. 2011 Dec 22;6:85. doi: 10.1186/1750-1326-6-8
Adult hippocampal neurogenesis and its role in Alzheimer's disease.


\bibitem{atlasAlzheimer}
	FeldMan H. H.,
	Atlas of Alzheimer's Disease.
	Informa Healthcare

\bibitem{MainADNI}
	Susanne G. Mueller, 
	Michael W. Weiner, 
	Neuroimaging Clin N Am. 2005 November ; 15(4): 869–xii.
	The Alzheimer’s Disease Neuroimaging Initiative
	
\bibitem{}
 Gorji, H. T.; Haddadnia, J. (2015-10-01). "A novel method for early diagnosis of Alzheimer's disease based on pseudo Zernike moment from structural MRI". Neuroscience. 305: 361–371. ISSN 1873-7544. PMID 26265552. doi:10.1016/j.neuroscience.2015.08.013.

\bibitem{}
 Zhu, Xiaofeng; Suk, Heung-Il; Shen, Dinggang (2014-10-15). "A novel matrix-similarity based loss function for joint regression and classification in AD diagnosis". NeuroImage. 100: 91–105. ISSN 1095-9572. PMC 4138265 Freely accessible. PMID 24911377. doi:10.1016/j.neuroimage.2014.05.078.

 \end{thebibliography}
