%%%%%%%%%%%%%%%%%%%%%%%%%%%%%%%%%%%%%%%%%%%%%%%%%%%%%%%%%%%%%%%%%%%
%%% Documento LaTeX 																						%%%
%%%%%%%%%%%%%%%%%%%%%%%%%%%%%%%%%%%%%%%%%%%%%%%%%%%%%%%%%%%%%%%%%%%
% T�tulo:	Plantilla de PFC/TFG/TFM
% Autor:  Ignacio Moreno Doblas
% Fecha:  2014-02-01
%%%%%%%%%%%%%%%%%%%%%%%%%%%%%%%%%%%%%%%%%%%%%%%%%%%%%%%%%%%%%%%%%%%
% Compilador: 	MiKTeX 2.9.
%	Modo:					PDFLaTeX.
% Entorno:			TeXnicCenter 2.0 Stable Beta 2.
%%%%%%%%%%%%%%%%%%%%%%%%%%%%%%%%%%%%%%%%%%%%%%%%%%%%%%%%%%%%%%%%%%%

% Pre�mbulo del documento.
%-----------------------------------------------------------------%
% Clase de documento: libro
\documentclass[12pt,a4paper]{book} % article, report, book.

% Pre�mbulo: paquetes, comandos, entornos, estilos y t�tulo de p�gina.
\input{A1.Preambulo.paquetes.tex}
\input{A2.Preambulo.commandos.tex}
\input{A3.Preambulo.entornos.tex}
\input{A4.Preambulo.estilodepagina.tex}

%Comando para crear glosario (index en ingl�s)
\makeindex


% Document body.
%-------------------------------------------------------------------%
\begin{document}

% Formato de documento hasta el cap�tulo 1 (sin incluirlo)
\frontmatter

%% T�tulo y autor del proyecto
%%Rellenar con el nombre real del autor, t�tulo, tutor y a�o.
%%El departamento del tutor se escribe una vez en el Resumen del PFC
%\renewcommand{\pfcauthorname}{Nombre del autor}
%\renewcommand{\pfctitlename}{T�tulo del proyecto}
%\renewcommand{\pfctutorname}{Nombre del tutor}
%\renewcommand{\pfcanno}{2014}

% Los tres ficheros siguientes solamente deben descomentarse
%  en el caso de ser un PFC del plan a extinguir.
% En los nuevos t�tulos de grado, no exiten la portada, la calificaci�n 
%  ni el Resumen del PFC.
% Estos tres documentos deben tomarse de la web de la ETSIT.
%
% Portada.
%\input{B1.Portada.tex}

% Hoja de calificaci�n
%\input{B2.Calificacion.tex}

% Resumen del proyecto (formulario)
%\input{B3.ResumenPFC.tex}

% Dedicatoria.
%%%%%%%%%%%%%%%%%%%%%%%%%%%%%%%%%%%%%%%%%%%%%%%%%%%%%%%%%%%%%%%%%%%
%%% Documento LaTeX 																						%%%
%%%%%%%%%%%%%%%%%%%%%%%%%%%%%%%%%%%%%%%%%%%%%%%%%%%%%%%%%%%%%%%%%%%
% T�tulo:	Dedicatoria
% Autor:  Ignacio Moreno Doblas
% Fecha:  2014-02-01
%%%%%%%%%%%%%%%%%%%%%%%%%%%%%%%%%%%%%%%%%%%%%%%%%%%%%%%%%%%%%%%%%%%
% Esta plantilla sirve para dedicatoria.

\cleardoublepage
\thispagestyle{empty} % No queremos mostrar ning�n encabezamiento ni pie de p�gina.

\begin{minipage}[c][\textheight][c]{\textwidth} %[pos][height][inner-pos]{width}
\raggedleft %\flushleft

A todas aquellas personas que han dedicado parte de su tiempo a mi desarrollo tanto personal como profesional.
\bigskip

\emph{Juan Carlos Martinez}

\end{minipage}

\blankpage

% Acr�nimos
%%%%%%%%%%%%%%%%%%%%%%%%%%%%%%%%%%%%%%%%%%%%%%%%%%%%%%%%%%%%%%%%%%%
%%% Documento LaTeX 																						%%%
%%%%%%%%%%%%%%%%%%%%%%%%%%%%%%%%%%%%%%%%%%%%%%%%%%%%%%%%%%%%%%%%%%%
% T�tulo:	Lista de acr�nimos
% Autor:  Ignacio Moreno Doblas
% Fecha:  2014-02-01
%%%%%%%%%%%%%%%%%%%%%%%%%%%%%%%%%%%%%%%%%%%%%%%%%%%%%%%%%%%%%%%%%%%

%Lista de acr�nimos 
\chapterbeginx{Acr�nimos}

\begin{acronym}[DLMS/COSEMM]
	
	
	\acro{ADNI}{Alzheimer's Disease Neuroimaging Initiative}
	\acro{AD}{\textit{Alzheimer Disease}}	
	\acro{CNN}{Red Neuronal Convolucional}	
	\acro{CSF}{Fluido Cerebro-Espinal}
	\acro{ETSIT}{Escuela T�cnica Superior de Ingenier�a de Telecomunicaci�n}
	\acro{GM}{Materia Gris}
	\acro{PFC}{Proyecto Fin de Carrera}
	\acro{PET}{Tomograf�a por Emisi�n de Positrones}
	\acro{TFG}{Trabajo Fin de grado}
	\acro{TFM}{Trabajo Fin de M�ster}
	\acro{MCI}{\textit{Mild Cognitive Impairment}}
	\acro{MRI}{Im�gen Resonancia Magn�tica}
	\acro{NC}{\textit{Normal Control}}
	\acro{ROI}{\textit{R�giones de Interes}}
	\acro{UMA}{Universidad de M�laga}
	\acro{WM}{Materia Blanca}
\end{acronym}

\chapterend

% Tabla de contenidos, figuras y tablas.
\input{B5.TableOfContents.tex}
\input{B5.TableOfFigures.tex}
\input{B5.TableOfTables.tex}

% Formato de documento durante los cap�tulos.
\mainmatter

% Pr�logo.
%%%%%%%%%%%%%%%%%%%%%%%%%%%%%%%%%%%%%%%%%%%%%%%%%%%%%%%%%%%%%%%%%%%
%%% Documento LaTeX 																						%%%
%%%%%%%%%%%%%%%%%%%%%%%%%%%%%%%%%%%%%%%%%%%%%%%%%%%%%%%%%%%%%%%%%%%
% T�tulo:	Pr�logo
% Autor:  Ignacio Moreno Doblas
% Fecha:  2014-02-01
%%%%%%%%%%%%%%%%%%%%%%%%%%%%%%%%%%%%%%%%%%%%%%%%%%%%%%%%%%%%%%%%%%%




\chapterbeginx{Resumen}
\par Este proyecto estudi� el uso de t�cnicas generativas aplicadas a la s�ntesis de neuroim�genes. La s�ntesis de neuroim�genes es un campo de un alto inter�s dado el elevado coste econ�mico  que tiene la adquisici�n de forma natural de este tipo de im�genes. Adem�s, uno modelo eficaz de s�ntesis reducir�a uno de los principales problemas que se encuentran los investigadores cuando se pretende generar modelos de diagnostico de Alzheimer asistido por computador. Este problema, denominado maldici�n de la dimensionalidad, es debido a la peque�a cantidad de muestras de neuroim�genes en comparaci�n a la alta dimensionalidad de dichas im�genes. 
\par Adem�s de evaluar la capacidad de s�ntesis, este proyecto analiz�  la capacidad de extracci�n de caracter�sticas del modelo generativo empleado, comprobando si la codificaci�n realizada es capaz de extraer aquellos patr�nes que permiten distinguir entre sujetos sanos y sujetos con Alzheimer.
\par En este trabjao se ha empleado el modelo basado en aprendizaje autom�tico conocido como autoencoder variacional. Este m�todo ha sido introducido recientemente y presenta la novedad  con respecto a modelos de autoencoder anteriores de generar un c�digo latente variacional. Se trata de un c�digo definido por una funci�n normal de distribuci�n, la cual dota de cierta libertad, dentro de lo m�rgenes de la distribuci�n, a la futura imagen regenerada a partir de dicho c�digo.
\par Se han dise�ado dos variantes basadas en el modelo del autoencoder variacional,  una haciendo uso de redes neuronales densas y otra basada en redes neuronales convolucionales. Se ha evaluado tanto la capacidad de codificaci�n as� como la capacidad de regeneraci�n de los dos modelos dise�ados. 
\par Los resultados obtenidos demuestran la capacidad del m�todo para obtener las caracter�sticas principales de las neuronim�genes ya que se consiguen valores en la clasificaci�n cercanos a los del estado actual del arte. No obstante, la s�ntesis de ima?enes realizada resulta inefectiva ya que las im�genes generadas por un mismo modelo para distinas imagenes de origin generan im�genes muy similares en las que no se aprecian los detalles iniciales de las im�genes.

 

\chapterend




\chapterbeginx{Summary}
\par Este proyecto estudia el uso de t�cnica generativas aplicadas a la sintes�s de neuroim�genes. La s�ntesis de neuroim�genes es un campo de un alto inter�s dado el elevado coste econ�mico  que tiene la adquisici�n de forma natural este tipo de im�genes. Adem�s, uno modelo eficaz de s�ntesis reducir�a uno de los principales problemas que se encuentran los investigadores a la hora de generar modelos CAD (Diagn�stico Asistido por Computador) conocido como maldici�n de la dimensionalidad.
\par A lo largo de este trabajo se analizar� el modelo basado en aprendizaje autom�tico conocido como autoencoder variacional. Este m�todo de autoencoder ha sido introducido recientemente y presenta la novedad  con respecto a modelos de autoencoder anteriores de generar un c�digo latente variacional. Se trata de un c�digo definido por una funci�n normal de distribuci�n, lo cual de dota de cierta libertad, dentro de lo m�rgenes de la distribuci�n, a la futura imagen regenerada a partir de dicho c�digo.
\par Se dise�aran dos variantes basadas en el modelo del autoencoder variacional,  una haciendo uso de redes neuronales densas y otra basada en redes neuronales convolucionales. Se evaluar� tanto la capacidad de codificaci�n as� como la capacidad de regeneraci�n de los dos modelos dise�ados. 
 


\chapterend

% Capitulos
\input{C1.Introduccion.tex}
\input{C2.Contexto.tex}
%%%%%%%%%%%%%%%%%%%%%%%%%%%%%%%%%%%%%%%%%%%%%%%%%%%%%%%%%%%%%%%%%%%
%%% Documento LaTeX 																						%%%
%%%%%%%%%%%%%%%%%%%%%%%%%%%%%%%%%%%%%%%%%%%%%%%%%%%%%%%%%%%%%%%%%%%
% T�tulo:		Cap�tulo 2
% Autor:  	Ignacio Moreno Doblas
% Fecha:  	2014-02-01
% Versi�n:	0.5.0
%%%%%%%%%%%%%%%%%%%%%%%%%%%%%%%%%%%%%%%%%%%%%%%%%%%%%%%%%%%%%%%%%%%
\chapterbegin{Fundamentos Te�ricos}
\label{chp:Utiliz}
%\minitoc

\par El trabajo realizado en este proyecto es englobado dentro de la tem�tica denomiada visi�n por computador, dado que los m�todos empleados se basan en la detecci�n de patr�nes sobre las im�genes dadas, en nuestro caso neuroim�genes. 

\par En el �mbito de la visi�n por computador cada imagen en s� misma es una muestra de miles dimensiones, cada uno de los pixeles. Un modelo generativo trata de capturar la relaci�n entre las multiples dimensiones de los datos. En nuestro caso el modelo empleado para capturar dichas relaciones es el Autoencoder Variacional.

\par Es por ello que este cap�tulo se centrar� en la exposici�n de este m�todo en primer lugar. Dado el VAE esta fundamentado en el aprendizaje profundo, se dedicar� la siguiente secci�n a las redes neuronales, haciendo especial hincapie a aquellas empleadas en este trabajo. Finalmente se expondr�n brevemente los m�todos estad�sticos usados de manera auxiliar a lo largo de este proyecto.  

\section{Autoencoder Variacional}

\par Este apartado est� dedicado a la exposici�n del  Autoencoder Variacional desde una perspectiva meramente te�rica  con objeto de mostrar los fundamentos y, en �ltima instancia, la capacidad de convergencia del m�todo, basada en una funci�n objetivo sobre la cual se puede aplicar descenso en gradiente estoc�stico. 

\subsection{Modelo de Variables Latentes}

\par

\begin{figure}[htp]
\centering
\includegraphics[scale=0.25]{images/ModeloVariablesLatentes.png}
\caption{Modelo gr�fico de variables latentes para el modelo generativo del VAE. $Z$ es el espacio de variables lo m�s similar posible a un distribuci�n normal $(N(0,I))$. El elemento $\theta$ es el conjunto de par�metros que aplicados de manera funcional sobre las variables latentes son capaces de generar el conjunto muestral $X$ }
\label{latentes_variables}
\end{figure}

\par A lo largo del entrenamiento o la caracterizaci�n de un modelo generativo, la parte mas complicada es la extracci�n de las dependencias entre las m�ltiples dimensiones. Son estas relaciones multidimensionales las que permiten generar muestras artificiales pertenecientes a clases distintas. Se denomina variable latente, a las unidades del modelo generativo capaces de discernir entre las distintas clases, esto es, capacitan al modelo para generar elementos diferenciados.



\par Un modelo generativo es representativo de un espacio muestral $(X)$ si para cada una de las muestras de dicho espacio $(x)$ hay al menos alguna configuraci�n de las variables latentes $(z)$ que genera un variable $(\hat{x})$ muy similar a la original. Formalmente, dada una funci�n  $f(z, \theta)$ parametrizada por un vector $\theta$ en un espacio $\Theta$ tal que:
\begin{center}
\begin{equation} \label{eq:space}
f : 	Z \times \Theta  \rightarrow  X  
\end{equation}
\end{center}

\subsection{Modelo Probabil�stico}
\par El objetivo es maximizar la probabilidad de cada $x$ de el espacio muestral de acuerdo con:
\begin{center}
\begin{equation} \label{eq:int_1}
P(X)  = \int_{}^{}P(X|z;\theta) 
\end{equation}
\end{center}
\par En la ecuaci�n \ref{eq:int_1}, $f(z;\theta)$ es reemplazada por la distribuci�n $P(X| z;\theta)$, la cual nos permite hacer expl�cita la dependencia de $X$ sobre $z$, debido a la probabilidad condicionada. 
La idea de detr�s de dicha expresi�n es principio de m�xima verosimilitud (ML, del ingl�s \textit{Maximum Likehood}), el cual indica que si el modelo es capaz de generar muestras del espacio $X$, entonces ser� probable que le modelo generativo construya muestras similares.

\par En el VAE, la funci�n de probabilidad $P(X|z;\theta)$ es las siguiente:

\begin{center}
\begin{equation} \label{eq:p_x_gausiana}
P(X|z; \theta)  = N(X| f(z;\theta), \sigma^{2}*I) 
\end{equation}
\end{center}

\par El uso de una distribuci�n gausiana nos permite emplear descenso en gradiente durante la optimizaci�n, con objeto de caracterizar el modelo. Esta caracterizaci�n permite incrementar $P(X)$, entendidada como la probabilidad global de generar alg�n tipo de muestra de dicho espacio. Esto no ser�a posible si esta funci�n de probabilidad fuera una delta de Dirac. Es importante notar que es fundamental disponener de una funci�n $P(X|z)$ que sea computable y continua sobre $\theta$.

\par Te�ricamente, para la mayor�a de los valores $z$, $P(X|z)$ ser� aproximadaente cero, y por lo tanto su contribuci�npara la estimaci�n de $P(X)$ ser� pr�cticamente nula.

\subsection{Funci�n Objetivo}

\par La principal idea en la que se fundamenta el VAE es en muestrear los valores de $z$ a partir de $X$, esto es, necesitamos una nueva funci�n $Q(z|X)$ que nos permita generar el conjunto de valores del espacio $Z$ a paritr de $X$. Esto nos reduce el espacio de $Z$ ya que, te�ricamente, este se ver� limitado en $Q(z|X)$. En �ltima instacia, esto nos permitir� estimar  $E[P(X|z)]$, siendo esta el valor esperado de la distribuci�n de probabilidad de los valores de X generados. 
\par La relacion entre $E(P(X|z))$ y $P(X)$ es uno de los fundamentos de los m�todos variacionales Bayesianos. Comencemos con la definici�n de la divergencia de KUllback-Leibler (\textit{KL}  o \textit{D}) entre una distribuci�n $P(z|X)$ y $Q(z)$:


\begin{center}
\begin{equation} \label{eq:KL_1}
D[Q(z)||P(z|X)] = E[log(Q(x)) - log(P(z|X))] 
\end{equation}
\end{center}

\par La expresi�n anterior, ecuaci�n  \ref{eq:KL_1}, es una medida no sim�trica de la similitud o diferencia entre las dos funciones de probabilidad $P(X|z) y Q(z)$. Dicha expresi�n mide diferencia (o el extra de informaci�n) entre un co�digo $P(x)$ y uno $Q(z)$. Aplicando la regla de Bayes sobre la expresi�n anterior conseguimos dejarlo en funci�n de $P(X)$ y $P(X|z)$:

\begin{center}
\begin{equation} \label{eq:KL_2}
D[Q(z)||P(z|X)] = E_{z}[log(Q(x)) - log(P(X|z)) - log(P(z)) + log(p(X))]  
\end{equation}
\end{center}

\par Ordenando la expresi�n anterior, y teniendo en cuenta que $log(p(X))$ no depende de $z$ por lo que puede salir del valor esperado:

\begin{center}
\begin{equation} \label{eq:KL_3}
log(p(X))- D[Q(z)||P(z|X)]  = E_{z}[log(P(X|z)]) - D[Q(z)||P(z)].  
\end{equation}
\end{center}

\par Llegados a este punto es importante notar que el espacio $X$ es fijo y por lo tanto tambi�n lo  es su funci�n de probabilidad $P(X)$. No obstante $Q(z)$ puede ser cualquier distribuci�n, siempre que nos permita generar $Z$ a partir de $X$.
\par Dado que en nuestro caso estamos intersados en inferir $P(X)$, es necesario generar una funci�n $Q$ dependiente sobre $X$ que permita que la divergencia $D[Q(z)||P(z|X)]$ sea peque�a, esto es, haya la menor perdida de informaci�n entre ambas distribuciones.

 \begin{center}
\begin{equation} \label{eq:KL_4}
log(p(X))- D[Q(z|X)||P(z|X)]  = E_{z}[log(P(X|z)]) - D[Q(z|X)||P(z)].  
\end{equation}
\end{center}

\par La expresi�n anterior, ecuaci�n \ref{eq:KL_4}, es la principal del VAE, por lo que es necesario examinarla detenidamente. Analizando cada t�rmino por separado:
\begin{itemize}
\item La expresi�n de la izquierda representa la cantidad que se prentende maximizar: $log(P(x))$, mas un t�rmino de error reperesentado por D[Q(z)||P(z|X)] que es la capacidad de generar $z$ a partir de $X$. Este t�rmino de error ser� disminuido si $Q$ es de alta capacidad. 
\par Se trata de maximizar $log(P(X))$ mientras simultaneamente $D[Q(z|X)||P(z)]$ se minimiza . El t�rmino de probabilidad $P(z|X)$ no es computable anal�ticamente, describe la distribuci�n de valores de $z$ que son capacades de generar $X$.
\item La expresi�n de la derecha es lo que se pretende optimar mediante el descenso en gradiente, dada una correcta seleccion de $Q(x)$.
\par Este segundo t�rmino fuerza la similitud entre $(Q(z|X))$ y$ P(X|z)$. Asumiendo que el t�rmino $Q(z|X)$ es de alta capacidad, tendr�mos que el t�rmino de divergencia KL ser� cercano a cero. En �ltima instancia, conseguiremos manejar de forma ana?itica $P(z|X)$ gracias a su similitud con $Q(z|X)$ 
\end{itemize}

\subsection{Optimizaci�n de la funci�n objetivo}

\par Con objeto de poder realizar el descenso en gradiente sobre la expresi�n de la derecha de la ecuaci�n \ref{eq:KL_4}, necesitamos definir de manera m�s exacta la forma de $Q(z|X)$. La elecci�n habiutual es la siguiente:

 \begin{center}
\begin{equation} \label{eq:OFU_1}
Q(z|X)  = N(z|\mu(X;\vartheta), \Sigma(X; \vartheta))  
\end{equation}
\end{center}

donde $\mu$ y $\Sigma$ son funciones determistas con una serie de par�metros $\vartheta$ (en las siguientes expresiones se omitir� $\vartheta$). Normalmente tanto $\mu$ como $\Sigma$ son implementados mediante redes neuronales y $\Sigma$ esta limitada a un funci�n diagonal, que permite facilitar los c�lculos.

\par El segundo t�rmino de la expersion \ref{eq:KL_4}, $D[Q(z|X)||P(z)]$, al ser una divergencia KL entre dos funciones de gausianas multivaradas queda definda por:


 \begin{align*}
 \begin{split}
D(N(\mu_{0}(X), \Sigma_{0}(X)|| N(\mu_{1}(X), \Sigma_{1}(X)))  = \\
\frac{1}{2}\left(tr(\Sigma_{1}^{-1}\Sigma_{0}) + (\mu_{1} - \mu_{0})^{T} \Sigma_{1}^{-1}(\mu_{1} - \mu_{0}) - k + log (\frac{det\Sigma_{1}}{det\Sigma_{0}})  \right)
\end{split} 
 \end{align*}  \label{eq:OFU_2}
 
donde $k$ es la dimensionalidad de la distribuci�n, la expresi�n queda de la siguiente manera:

 \begin{align}
 \begin{split}
D[N(\mu(X), \Sigma(X)) || N(0, I))] = \\
\frac{1}{2} \left( tr(\Sigma(X)) + (\mu(X))^{T}(\mu(X) - k -log(det(\Sigma(X)))) \right)
\end{split} 
 \end{align}  \label{eq:OFU_3}

\par El primer t�rmino de la expresion \ref{eq:KL_4}, $E_{z}[log(P(X|z))]$, es algo m�s complicado de determinar, aunque a priori se podr�a estimar usando un n�mero suficentes de $z$ y aplicando al funci�n $f$ asociada a $P(X|z)$, aunque esto ser�a tremendamente costoso computacionalmente. 
\par En  su lugar, se aplica un procedimiento denomidado Descenso en Gradiente Estoc�stico (SGD, del ingl�s \textit{Stochastic Gradient Descent}), que se basa en tomar �micamente un valor de $z$ aplicarlo sobre $P(X|z)$, por lo que se obtendr�a una aproximaci�n de $E_{z}[log(P(X|z))]$. Durante este proceso, estamos tomando como referencia cada una de las muestras $X$ de un conjunto de datos $D$ a la hora de estimar el error. Teniendo en cuento esto, la ecuaci�n completa que se pretnede optimizar es:

 \begin{align}
 \begin{split}
E_{X}[log(P(X)) - D[Q(z|X) || P(z| X)]] = \\
E_{X}[E_{z}[log(P(X|z))] - D[Q(z|X) || P(z)]]
\end{split} 
 \end{align}  \label{eq:OFU_4}

\par Tomando el gradiente de la expresi�n anterior, reducimos la expresi�n a los valores internos de las esperanzas. Adem�s, podemos tomar un �nico valor de $X$ y un �nico valor de $z$ de la distribuci�n $Q(z|X)$, lo que no nos permite hacer computable el gradiente de la siguiente forma: 

 \begin{center}
\begin{equation} \label{eq:OFU_5}
log(P(X|z)) - D[Q(z|X) || P(z)]. 
\end{equation}
\end{center}

\par No obstante hay un problema significativo en la ecuaci�n \ref{eq:OFU_4} ya que $E_{z}[log(P(X|z))]$
depende de los par�metros de $P$ y tambi�n de los valores de $Q$. Esto es problem�tico a la hora de realizar el descenso en gradiente, quedando resuelto con lo que se conoce como "Truco de Reparametrizaci�n".

\subsection{El truco de Reparametrizaci�n} \label{sec:repa_1}

\begin{figure}[htp]
\centering
\includegraphics[scale=0.4]{images/trucoReparametrizacion.png}
\caption{(Izquierda) Modelo de VAE sin Truco de Reparametrizaci�n. (Derecha) Modelo de VAE con Truco de Reparametrizaci�n}
\label{fig:repara_trick}
\end{figure}

\par Para garantizar el correcto funcionamiento del VAE es necesario que la funci�n codificadora ($f$) asociada a $Q(z|X)$ generare un conjunto $Z$ capaz de ser decodificado por la funci�n generadora ($g$) asociada a $P(X|z)$.
\par Analizando el problema desde otra perspectiva, tomando como referencia el diagrama de izquierda de la figura \ref{fig:repara_trick}. El paso hacia delante\footnote{En el �mbito de las redes neuronales se denomina paso hacia delante (del ing�s \textit{forward pass}) al proceso incial de evaluar la salida generada a partir de una determinada entrada. En nuestro caso la entrada es $X$ y la salida $f(z)$, siendo la evaluaci�n realizada $||X - f(z)||$} funciona de manera de correcta y es de esperar (si los par�metros est�n correctamente entrenados) que la salida produzca un salida acertada de manera general. 
\par No obstante, es necesario realizar el paso hacia atr�s\footnote{En el �mbito de las redes neuronales, el paso hacia atr�s (del ingl�s \textit{backpropagation}) hace referencia al proceso de evaluar el gradiente en cada uno de los elementos del sistema, tomando como referencia que el error se?a el determinado del paso hacia delante} teniendo que determinar el gradiente sobre la funci�n $Q(z|X)$ encargada de generar $z$, pero este modelo de generaci�n esta basado en el mapeo sobre una distribuci�n gausiana, lo cual es una funci�n no continua.

\par La soluci�n a este problema se denomica truco de reparametrizaci�n (del ingl�s \textit{reparameterization trick}) el cual se basa en trasladar el mapeo sobre la distribuci�n gausiana a una capa de entrada. 
\par Dados $\mu_{X}$ y  $\Sigma_{X}$, media y convarianza respectivamente de $Q(z|X)$, podemos mapear $N(\mu_{X}, \sigma(X))$ tomando un valor de la funci�n Normal ($\epsilon \sim N(0,I)$) y aplicando la siguiente expresi�n:
\begin{center}
\begin{equation} \label{eq:RT_1}
z = \mu(X) + \Sigma(X)^{1/2} * \epsilon. 
\end{equation}
\end{center}

\par Por lo tanto la funci�n final, la cual queda representada en el diagrama de la derecha de la figura  \ref{fig:repara_trick}, sobre la que se aplica el gradiente es la siguiente:

 \begin{align}
 \begin{split}
E_{X\sim Z}\left[E_{\epsilon\sim N(0,I)}[log(P(X|z = \mu(X) + \Sigma^{1/2} * \epsilon))] - D[Q(z|X) || P(z)]\right].
\end{split} 
 \end{align}  \label{eq:RT_2}

\par Cabe notar que ninguna de las esperanzas son con respecto a las distribuciones caracter�sticas del sistema (ni $P(X|z)$ ni $Q(z|X)$) lo que nos permite realizar el gradiente sin ning�n problema sobre los elementos contenidos dentro de los valores esperados, ya que el gradiente es la derivada sobre los par�metros funcionales de estas distribuciones.
\par Por lo tanto dado un valor de $X$ y $\epsilon$ la funci�n \ref{eq:RT_2} ser� continua y determinista sobre los par�metros de $P$ y $Q$, lo cual nos permite realzar el paso hacia atr�s de manera eficaz. 
 
 
\subsection{Interpretaci�n de la funci�n objetivo} 


\begin{figure}[htp]
\centering
\includegraphics[scale=0.25]{images/FO_interpretacion_2.png}
\caption{Traducir }
\end{figure}


\begin{figure}[htp]
\centering
\includegraphics[scale=0.25]{images/FO_interpretacion_1.png}
\caption{Traducir }
\end{figure}


 
\subsection{Codificaci�n y Decodificaci�n}

\par La eficacia y tratabildad del m�todo reside en la asunci�n de que $Q(z|X)$, la funci�n coficadora, puede ser modelada como una gaussiana con una media determinada $\mu(X)$  y varianza $\Sigma(X)$, por otro lado es necesario que $P(X)$ converja de manera eficaz a la distribuci�n real de los datos del espacio $D$. Estas condiciones solo son superadas si y solo si $D[Q(z|X) || P(z|X)]$ es cercana a cero.
\par Es por ello necesario una funci�n $Q$ de alta capacidad, lo cual puede llevarnos a modelos complejos. Los modelos basados en funciones usados en los VAE son las redes neuronales


\begin{figure}[htp]
\centering
\includegraphics[scale=0.7]{images/decodificador,codificador.png}
\caption{Esquematizaci�n simple de las funciones del Codificador y el Decodificador en el VAE}
\label{fig:CD_1}
\end{figure}

\par El codificador es una red neuronal. Su entrada es el dato $X$ y su salida es la representaci�n latente $z$. Representa la distribuci�n de probabilidad $Q(z|X)$, y esta determinada por el conjunto de par�metros y pesos de la red neuronal asociada. Denominaremos a la funci�n encargada de la codificaci�n $q_{\theta}(z|x)$
\par El codificador se identifica a menudo con el proceso de reducci�n de la dimensionalidad de $x$ a $z$. Cabe notar que el codificador tiene asociadas dos funciones, una encargada de obtener la media $q_{\mu}(X)$  y otro la varianza $q_{\Sigma}(X) $ del espacio latente. Para la obtenci�n final de $z$ se ha de aplicar  el truco de reparametrizaci�n, ver seccion \ref{sec:repa_1}, con respecto a los valores $\Sigma$ y $\mu$ obtenidos anteriormente.

\par El decodificador es otra red neuronal. Su entrada es la variable del espacio latente $z$ y su salida es la reconstrucci�n del dato inicial $X$. Denominaremos a la funci�n encargada de la decodificaci�n $p_{\phi}(x|z)$, donde $\phi$ son el conjunto de par�metros y pesos que definen la red neuronal.

\par El hecho de que ambas funciones est�n basadas en redes neuronales hace el aprendizaje profundo sea una parte primordial del VAE. T�picamente los formatos de redes neuronales aplicados en este sistema son dos; redes neuronales densas (DNN)  o redes neuronales convolucionales (CNN).

\newpage
\section{Redes Neuronales}
\par Las Redes Neuronales permiten generar funciones complejas no lineales gracias a su capacidad inherente de aprendizaje con el proceso denomiado propagaci�n hacia atr�s, que permiten ajustar los pesos de las distintas unidades o neuronas del sistema. 
\par Dada la complejidad del �mbito del aprendizaje profundo, en las siguiente seciones se pretenden exponer las ideas fundamentales para comprender el comportamiento de las funciones de codificacion y decodificaci�n del VAE, sin entrar en explicaciones excesivamente te�ricas sobre los fundamentos de las redes neuronales. 
\par Es por ello que en primer lugar se expondr� el modelo de redes neuronales densas, aprovechando para exponer de manera somera algunos conceptos de redes neuronales, como son el concepto de funciones de activaci�n o el proceso de propagaci�n hacia atr�s. 

\par Seguidamente se expondr� el otro modelo de aprendizaje profundo utilizado en este trabajo que son las redes neuronales convolucionales, explcando por que son ideales para la captura de patrones sobre im�genes. 

\subsection{Red Neuronal Densa}
\par Este modelo constituye el paradigma b�sico de redes neuronales. Fundamentado en el est�ndar de neuronal artificial  seg�n los principios descritos Rumelhart y McClelland en 1986 \cite{DNN_1}. Siguiendo dichos principios, la i-�sima neurona artificial  consiste en:

\begin{figure}[!hb]
\centering
\includegraphics[scale=0.5]{images/DNN_1.png}
\caption{Sistema global de proceso de una red neuronal}
\label{fig:CD_1}
\end{figure}


\begin{itemize}

\item Un conjunto de entradas $x_j$ con un conjunto de pesos sin�pticos asociados $w_{ij}$, con $j=1,2...n$
\item Una regla de propagaci�n $h_i$ a definida partir del conjunto de entradas  y de los pesos sin�pticos. Normalmente la regla de propagaci�n utilizada el producto lineal entre los pesos sin�pticas y las entradas. Esto es:
\begin{center}
\begin{equation}\label{eq:DNN_1}
h_i(x_1,.....,x_m, w_{i1}....,w{in} = \sum_{i=1}^{n}w_{ij}*x_{j}
\end{equation}
\end{center}

\item Una funci�n de activaci�n, la cual representa simult�neamente la salida de la neurona y su estado de activaci�n. Denotando por $y_i$ dicha funci�n de activaci�n:

\begin{center}
\begin{equation} \label{eq:DNN:2}
y_i = f_i(h_i) = f_i(\sum_{j=0}^{n}w_{ij}x_{j})
\end{equation}
\end{center}  

\end{itemize} 

\subsubsection{Funci�n de Activaci�n}
\begin{figure}[!b]
\centering
\includegraphics[scale=0.45]{images/functions_activations.png}
\caption{Principales funciones de activaci�n.}
\label{fig:CD_1}
\end{figure}
\par La eleccion de la funci�n de activaci�n constituye una parte determinante en el dise�o de redes neuronales, dado que afectar� en gran medida al a capacidad de decisi�nd de la red y la rapidez con que la red sea capaz de converger durante el entrenamiento \cite{FA_1}. 
\par En general el principal requerimiento sobre estas funciones es que sean capaces de respetar el proceso del propagaci�n hacia atr�s, no provocando que el gradiente se haga cero lo cual repercutir�a negativamente en el proceso del descenso en gradiente. Este es uno de los problemas asociadas a la clasica funci�n sigmoide, dado que para valores de $x$ ampliamente negativos o positivos, provoca que el gradiente sea cero\footnote{Este efecto es conm�nmnte denominado como saturaci�n}, interrumpiendo el descenso en gradiente para la neurona en cuesti�n y, por tanto, la optimizaci�n de sus pesos sin�pticos.



\begin{figure}[!hb]
\centering
\includegraphics[scale=0.45]{images/lrelu_2.png}
\caption{(izquierda) Funci�n de activaci�n \textit{Relu}. (Derecha) Funci�n de activaci�n \textit{leakyRelu.} }
\label{fig:CD_2}
\end{figure}

\par Actualmente la funci�n de activaci�n m�s utilizada es la unidad lineal de rectificaci�n \cite{FA_2} (ReLu del ingl�s \textit{Rectifier Linear Unit}), representada en la figura \ref{fig:CD_1}. No obstante, otro tipo de funci�n de activaci�n basada en la anteriomente expuesta  denominada unidad lineal de rectificacion con p�rdidas (leakyRelu) ha ganado peso en el �mbito. La �nica diferencia entre ambas funciones es la capacidad de la \textit{leakyRelu} de no hacer nulo el gradiente para valores negativos, ver figura \ref{fig:CD_2} para apreciar esta diferencia. En este proyecto han sido utilizadas tanto la funcion Sigmoide como la funci�n \textit{leakyRelu}


\subsubsection{Topolog�a de Conexionado}
\begin{figure}[!h]
\centering
\includegraphics[scale=0.4]{images/capas.png}
\caption{Esquema de una red neuronal densa de una sola capa oculta }
\label{fig:DNN_4}
\end{figure}


\par Otro concepto determinante en el comportamiento de las redes neuronales es la topolog�a empleada, esto es, el patron de conexionado de una red neuronal. En una red neuronal artificial los nodos se conectan entre s�, siendo este conjunto de conexiones internas junto con los pesos sin�pticos lo que determina el comportamiento de la red y, en �ltima instancia, la funci�n asociada  a la red. 
\par Las unidades neuronales suelen agruparse en lo que se denominan capas. La uni�n de dos o m�s capas constituyen una red neuronal. Se distinguen tres tipos de capas: de entrada, de salida y ocultas. Una capa de entrada esta compuesta por las neuronas que reciben las se�ales. Una capa de salida est� constituida por el conjunto de neurones que proporcionan la respuesta de la red. Las capas ocultas no tienen conexionado con el exterior. A m�s capas Socultas m�s capacidad de aprendizaje tendr� el sistema, aunque el tiempo necesario para su optimizaci�n aumentar� considerablemente. 


\subsubsection{Propagaci�n Hacia Atr�s}

Se denomina propagaci�n hacia atr�s al proceso empleado para el entrenamiento de las redes neuronales. Este entrenamiento tiene como objetivo el ajuste de los pesos sin�pticos de la red. Se considera un buen ajuste de pesos aquel que minimiza el error a la salida de una red \cite{DNN_training}. De manera breve los principales pasos de este proceso de entrenamiento son:



\begin{figure}[b!]
\centering
\includegraphics[scale=0.30]{images/entrenamiento.png}
\caption{Representaci�n esquem�tica del proceso de entrenamiento de una red neuronal }
\label{fig:DNN_5}
\end{figure}

\begin{itemize}
\item Inicializaci�n. Se asigna un valor por defecto a los distintos pesos. Se considera un paso determinante, puesto que una mala inicializaci�n puede implicar la saturaci�n de los gradientes en los nodos.
\par Los siguientes pasos constituyen un proceso iterativo, durante el cual se ir� minimizando progresivamente el error asociado a la salida de la red. 
\item Paso hacia delante  (\textit{Fast Forward}). Se comprueba el comportamiento de la red, se calcula la salida de la red para un conjunto de muestras de entrada. 
\item Estimaci�n del error de salida. Dada una salida, se eval�a la diferencia con respecto a la salida esperada seg�n las muestras de entrada. 
\item Se realiza la propagaci�n hacia atr�s. Dado el error a la salida se realizan el conjunto de derivadas necesarias recorriendo desde la salida hacia la entrada la red, identificando el comportamiento del gradiente del error con respecto a los diferentes pesos de la red.
\item Se modifican los pesos en funci�n del gradiente previamente calculado. 

\end{itemize}



\newpage
\subsection{Red Neuronal Convolucional}

\par Las redes convolucionales (CNN, del ingl�s \textit{Convolutional Neural Networks}) son una categor�a de redes neuronales que se consideran un m�todo altamente eficaz en �reas como el reconocimiento de im�genes \cite{CNN_1}\cite{CNN_2}.
\begin{figure}[]
\centering
\includegraphics[scale=0.50]{images/convNet.png}
\caption{Red convolucional \textit{LeNet5} }
\label{fig:CNN_1}
\end{figure}


\par Este modelo fu� introducido en 1989 \cite{CNN} por Yann leCunn, la red de este trabajo fue denominada \textit{LeNet5}. Dicha red se puede observar en la imagen \ref{fig:CNN_1}
\par Las redes convoluciones suelen ser aplicadas a las im�genes. Cada imagen puede ser representada por una matriz de n�meros s� se trata de una imagen en escala de grises, o por tres matrices s� es una imagen a color. Es esta propiedad de las im�genes donde cada dimensi�n, es decir cada pixel, queda definida espacialmente con respecto al resto de dimensiones, lo que convierte a las im�genes en las muestras ideales para este tipo de red. 
\par Se asume que los conjuntos de pixeles vecions formar�n unas caracter�sticas m�s significativas que s� tomaramos grupos sin tener en cuenta su disposici�n espacial

a capa anterior lo que se emplea es el operador de convoluci�n.
\begin{figure}[b]
\centering
\includegraphics[scale=0.60]{images/conNet_2.png}
\caption{Ejemplo de aplicaci�n del operador convoluci�n sobre una imagen. Seleccionada una regi�n de la imagen cuyas dimensiones son las mismas que las del kernel seleccionado, se aplica el producto pixel a pixel entre dicha regi�n y los pesos propios del kernel. La suma de estos productos se almacena en la imagen de salida, respetando la ubicaci�n espacial de la regi�n evaluada. }
\label{fig:CNN_2}
\end{figure}

\par Este tipo de redes derivan su nombre del operador de red convoluci�n cuyo objetivo es extraer caracter�sticas de las im�genes preservando la relaci�n espacial entre pixeles.  

\par Dada una imagen bidimensional \textit{I} y una  matriz $K$ de dimensiones $h\times w$ (denominada kernel de convoluci�n) la cual es capaz de extraer alg�n tipo de caracter�stica relevante. La operaci�n de convoluci�n se puede representar como:  


\par Formalmente, se puede expresar como: 
\begin{center}
\begin{equation} \label{eq:CNN_2}
(I*K):{xy} = \sum_{i=1}^{h}\sum_{j=1}^{w} K_{ij}I_{I_{x+i-1},u+j-1}
\end{equation}
\end{center}  
 
\par A diferencia de las redes neuronales convencionales en las redes convolucionales los datos a la entrada y entre el conexionado de capas se agrupan en 3 dimensiones: ancho, alto y profundidad. En este caso nos referimos a "profundidad" por capa no a la profundidad de la red, lo cual se refiere al n�mero de capas  de la red en cuesti�n. Por ejemplo, dada una imagen de entrada de tres canales (los tres canales de color) de 32x32 pixeles, la agrupaci�n de los datos en la capa de entrada ser� 32x32x3. Ver figura \ref{fig:CNN_3}
\par Otra diferencia con respecto a la redes neuronales cl�sicas es que las unidades de una capa solo est�n conectadas a un espacio reducido de unidades de la capa inmediatamente anterior. 

\begin{figure}[t]
\centering
\includegraphics[scale=0.60]{images/convNet_3.png}
\caption{(Izquierda) Modelo clasico de redes neuronales. (Derecha) Modelo de Red Convolucional. Los datos son reagrupados en 3 dimensionados como se puede observar en una de las capas. Cada una de las capas tiene como entrada una imagen 3D y tiene como salida otra imagen 3D. La capa roja representa la capa de entrada por lo que la altura y la anchura son las dimensiones de la imagen y la profundidad son el n�mero de canales}
\label{fig:CNN_3}
\end{figure}

\par Las redes neuronales convolucionales se fundamentan en tres principios b�sicos que son los campos receptivos locales, los pesos compartidos y el empleo de agrupaciones o \textit{pooling}

\subsubsection{Filtros Locales}

\par En una red neuronal densa, esto es, una red totalmente conectada como la de imagen \ref{fig:CNN_3} las entradas se interpretan como un conjunto "vertical" de unidades. Sin embargo, en un red convolucional es preferible organizar las unidades de entrada en forma bidimensional.
\begin{figure}[t]
\centering
\includegraphics[scale=0.60]{images/conNet_4.png}
\caption{Representaci�n de la conectividad local en una red neuronal}
\label{fig:CNN_4}
\end{figure}

\par Las capas consecutivas estar�n conectadas entre s�, pero cada unidad de una capa oculta estar� conectada solo a un conjunto de unidades de la capa inmediatamente anterior.
\begin{figure}[b]
\centering
\includegraphics[scale=0.60]{images/ConvNet_6.png}
\caption{Desplazamiento del campo de recepci�n}
\label{fig:CNN_5}
\end{figure}


\par Se denomina filtro local a la ventana que se aplica a las diferentes regiones seleccionables de la imagen, cada una de estas r�giones seleccionables estan conectadas a una �nica unidad de la siguiente capa oculta. A este t�rmino a menudo nos referimos como kernel. Este filtro se desplazar� por toda la imagen, realizando el proceso de convoluci�n por toda ella, ver imagen \ref{fig:CNN_5}. Es esto lo que permite extraer caracter�sticas de manera local por toda la imagen
\par Normalmente el desplazamiento se hace pixel a pixel aunque es posible aumentar el n�mero de pixeles por desplazamiento. Este hiperpar�metro se denomina generalmente \textit{stride}. En este trabajo se ha utilizado un valor de dos. Otro concepto a tener en cuenta es que por lo general hay varios tipos de fitros para la extracci�n de caracter�sticas en las diferentes capas.  

\subsubsection{Pesos Compartidos}

\par Cada uno de los filtros de recepci�n ser�n aplicados a toda la imagen con el mismo peso para todas las diferentes regiones. Esto significa que el patr�n de selecci�n de caracter�sticas ser� el mismo, por lo que las neuronas de la siguiente capa detectar�n el mismo tipo de caracter�stica. 
\par El punto anterior se fundamente en que generalmente un patr�n  de una parte de la imagen es probable que se repita en otra parte de la imagen dada la propia naturaleza de las im�genes. 
\par Con objeto de no limitar cada capa a la extracci�n de un tipo de caracter�sticia se aplican numeros filtros en cada una de las capas de convoluci�n. Gracias a esto se consiguen extraer distintos tipos de patrones. 

\begin{figure}[t]
\centering
\includegraphics[scale=0.40]{images/conv_net7.png}
\caption{Representaci�n de la extracci�n de varias caracter�sticas con varios filtros}
\label{fig:CNN_6}
\end{figure}
\par Una de la ventajas del uso de pesos compartidos es que permite reducir el n�mero de par�metros de la red. 


\begin{figure}[t]
\centering
\includegraphics[scale=0.50]{images/convNet_7.png}
\caption{Representaci�n del proceso de agrupamiento (\textit{pooling})}
\label{fig:CNN_}
\end{figure}

\subsubsection{Agrupamiento}

\par Otro tipo de capa caracter�stica de las redes convolucionales son las capas de agrupamiento o \textit{pooling}.  Esta capa tiene como objetivo reducir el n�mero de datos generado, realizando una estimaci�n del valor m�s importante de una determinada region. Esto permite reducir progresivamente el tama�o de la imagen. Este proceso de agrupamiento se aplica individualmente a cada una de las imagnenes generadas por cada filtro.  
\par No obstante, esta funcionalidad no ha sido utilizada en el modelo generado en este trabajo dado que actualmente la librer�a empleada (\textit{TensorFlow})  no tiene implementada esta operaci�n para im�genes 3D. 




\newpage
\section{Herramientas Complementarias}
\subsection{M�quina de Vectores de Soporte}
\subsection{Validaci�n Cruzada (K-fold)}
\subsection{M�tricas de Validaci�n}







































\chapterend{}
%%%%%%%%%%%%%%%%%%%%%%%%%%%%%%%%%%%%%%%%%%%%%%%%%%%%%%%%%%%%%%%%%%%
%%% Documento LaTeX 																						%%%
%%%%%%%%%%%%%%%%%%%%%%%%%%%%%%%%%%%%%%%%%%%%%%%%%%%%%%%%%%%%%%%%%%%
% T�tulo:		Cap�tulo 2
% Autor:  	Ignacio Moreno Doblas
% Fecha:  	2014-02-01
% Versi�n:	0.5.0
%%%%%%%%%%%%%%%%%%%%%%%%%%%%%%%%%%%%%%%%%%%%%%%%%%%%%%%%%%%%%%%%%%%
\chapterbegin{Trabajo Realizado} 

\label{chp:Utiliz}
%\minitoc
\section{Estudio basado en r�giones cerebrales}
\par Uno aspecto clave de este trabajo consiste en la divisi�n del cerebro en diferentes �reas con objeto de ser caracterizadas de manera aislada, y en �ltima instancia, poder generar cada una de las �reas o r�giones por separado. Para llevar a cabo esta separaci�n se ha usado en atlas AAL (del ingl�s \textit{Automated Anatomical Labeling}) \cite{AAL} que define un total de 116 r�giones las cuales se corresponden con las diferentes �reas anat�micas. Este atlas permite obtener los v�xeles asociados a cada regi�n de manera normalizada.  

 \par Aunque  esta aproximaci�n tiene la ventaja de permitirnos caracterizar las r�giones por separado,
el principal motivo por el que hemos el alto coste computacional que lleva asociado el  uso de redes de aprendizaje profundo cuando son aplicadas en datos de alta dimensionalidad, como es nuestro caso. Otro problema derivado el amplio tiempo necesario para la caracterizaci�n de los par�metros de lared. 
\par El uso de una aproximaci�n basada en r�giones nos permite reducir de forma considerable el n�mero de voxeles a caracterizar por cada red neuronal y por lo tanto reducir los tiempos de caracterizaci�n y los costes computacionales

\par Dentro de las 116 r�giones en las que se han dividido las neuroim�genes,  normalmente aquellas a las que se les atribuye que aportan informaci�n sobre la detecci�n del AD se las denomina R�giones de Inter�s (ROI, del ingl�s \textit{Regions of Interest}).
\par Se ha almacenado el atlas AAL junto a cada uno de las distintas modalidades de im�genes empleadas que son las MRI y las PET. Esto nos permite extraer las r�giones indicades mediante el Atlas de las distintas im�genes, lo cual nos posibilita hacer el procesado de r�giones de manera independiente. En la imagen se \ref{fig:Atlas} muestra un ejemplo de imagen PET y MRI con su atlas correspondiente al lado. 

\begin{figure}[htp]
\centering
\includegraphics[scale=0.55]{images/img_4.png}
\caption{MRI image (a), MRI atlas (b), (c) PET image
and (d) PET atlas (same slice is shown in MRI and PET
images)}
\label{fig:Atlas}
\end{figure} 
 


\section{Tratamiento de Neuroim�genes}
\par En todo proceso de caracterizaci�n de muestras o de aprendizaje estad�stico un aspecto esencial es aplicar un tratamiento y procesado efectivo de las muestras previo al algoritmo principal, ya que si las muestras no son correctas o se producen irregularidades en su tratamiento previo se estar� avocado a unos resultados incorrectos (por muy bien elaborado que est� el algoritmo principal).

\par En primer lugar se explicar�n las caracter�sticas demogr�ficas de las neuroimagenes empleadas. Posteriormente se expondr� de manera resumida el procesado de las im�genes realizado por el grupo de investigaci�n de And?es Ortiz.

\subsection{Fuente de Datos}
\par Las neuroim�genes empleadas en este trabajo pertenecen a la iniciativa ADNI. Se han empleado tanto im�genes MRI como im�genes PET. Se disponen de 229 im�genes MRI de sujetos NC y 188 de sujetos de AD. Por otro lado, en el caso de las im�genes PET se disponen de 70 im�genes de sujetos AD y 68 de sujetos NC. La distribuci�n demogr�fica de los sujetos se puede observar en las tablas \ref{tb:Demografia_1} y \ref{tb:Demografia_2}

\begin{table}[b]
\centering
\begin{tabular}{ c | c | c | c | c } 
 Diagnosis & Number & Age &  Gender (M/F) & MMSE  \\ \hline
 Control & 68  &  75.81 $\pm$ 4.93 & 43/25 & 29.06 $\pm$ 1.08 \\ \hline
 AD & 70  & 73.06 $\pm$ & 46/24  & 22.84 $\pm$ 2.61  

\end{tabular}
 \caption{Datos Demogr�ficos Im�genes MRI}
\label{tb:Demografia_1}

\end{table}

\subsection{Procesado Previo}

\par Las im�genes PET y MRI de la base de datos ADNI han sido espacialmente normalizadas de acuerdo con el modelo de morfometr�a basada en v�xeles T1 \cite{VBM} (\textit{VBM-T1} del ingl�s \textit{Voxel-Based Morphology}), con objeto de garantizar que cada v�xel corresponde con la misma posici�n anat�mica en cada una de las neuroim�genes. Posteriormente las im�genes MRI fueron redimensionadas a $121\times145\times121$ v�xeles con un tama�o de v�xel de 1.5 mm ,(Sagital) $\times$ 1.5 mm (Coronal) $\times$ 1.5 mm (axial). Por otro lado, las im�genes PET fuer�n redimensionadas a $79\times95\times68$  con un tama�o de v�xel de 3 mm (Sagital) $\times$ 3 mm (Coronal) $\times$ 2 mm (axial).
\par Las im�genes MRI son tratadas de manera diferente que las PET ya que son segmentadas en tejido de materia gris (GM del ingl� \textit{Grey Matter}) y tejido de materia blanca (WM del ingl�s \textit{White Matter}) aplicando la herramienta SPM \cite{VBM_2}\cite{VBM_3} de normalizaci�n espacial. Este proceso es  capaz de generar informaci�n sobre al distribuci�n del tejido de GM, de WM o de fluido Cerebro-Espinal (CSF del ingl�s \textit{Cerebrospinal Fluid}) en las neuroim�genes, quedando caracterizado por una probabilidad de pertenencia  para cada uno de los tejidos de rango [0, 1]. 

\begin{figure}[htp]
\centering
\includegraphics[scale=0.55]{images/ejemplo_MRI.png}
\caption{Muestra de neuroimagen MRI segmentada. (zquierda) MRI WM image. (Derecha) MRI GM image}
\label{fig:Atlas}
\end{figure} 


\par Por otro lado, las im�genes PET son normalizadas con respecto al nivel de intensidad. Este nivel m�ximo de intensidad se toma a partir del nivel medio del 1\%  de los v�xeles con mayor activaci�n del cerebelo \cite{PET_norm}, dado que esta regi�n cerebral es considerada con activaci�n constante. Este proceso de normalizaci�n  permite la homogeanizaci�n de los niveles entre los v�xeles permiti�ndonse las posterior comparaci�n entre v�xeles. 


\begin{figure}[htp]
\centering
\includegraphics[scale=0.4]{images/ejemplo_PET.png}
\caption{Muestra de neuroimagen PET normalizada}
\label{fig:Imagen PET normalizada}
\end{figure} 


\subsubsection{Preselecci�n de V�xeles}
\par La preseleci�n de v�xeles se ha aplicado a cada modalidad de imagen con objeto de eliminar los v�xeles poco significativos. Esto nos permite reducir el alto coste computacional asociado a la alta dimensionalidad de las im�genes. Esta preselecci�n de caracter�sticas ha sido realizada mediante el \textit{t-test de Welch} sepradamente sobre cada tipo de imagen. 

\par El \textit{t-test Welch} permite evaluar la diferencia entre la media de dos espacios muestrales, en nuestro caso NC y AD, cuando las varianzas no son iguales y puede ser calculado usando la siguiente expresi�n:


\begin{center}
\begin{equation} \label{eq:t-test}
I^{t}  = \frac{I_{NC}^{\mu} - I_{AD}^{\mu}}{ \sqrt{\frac{I_{NC}^{\sigma}}{N_{NC}} + \frac{I_{AD}^{\sigma}}{N_{AD}}  }}
\end{equation}
\end{center}

\par donde $I_{NC}^{\mu}$ y $I_{AD}^{\mu}$ son las medias de las imagenes de los sujetos NC y AD respectivamente, mientras que  $I_{NC}^{\sigma}$ y $I_{AD}^{\sigma}$ son las varianzas de las imagenes y $N_{NC}, N_{AD}$ son el n�mero de muestras NC y AD. Las imagenes de medias  $I_{NC}^{\mu}$ y $I_{AD}^{\mu}$ se calculan como:

\begin{center}
\begin{equation} \label{eq:t-test-means}
I_{NC}^{\mu} = \frac{1}{N_{NC}}\sum_{j=1}^{N_{NC}}I_{j}, \quad I_{AD}^{\mu} = \frac{1}{N_{AD}}\sum_{j=1}^{N_{AD}}I_{j},
\end{equation}
\end{center}

mientras que las varianzas de las im�genes $I_{NC}^{\sigma}$ y $I_{AD}^{\sigma}$ son calculadas mediante:

\begin{center}
\begin{equation} \label{eq:t-test-vars}
I_{NC}^{\sigma} = \frac{1}{N_{NC}}\sum_{j=1}^{N_{NC}}(I_{j} - I_{j}^{\mu})^2, \quad I_{AD}^{\sigma} = \frac{1}{N_{AD}}\sum_{j=1}^{N_{AD}}(I_{j} - I_{j}^{\mu})^2
\end{equation}
\end{center}

\par En la ecuaci�n \ref{eq:t-test} el t�rmino $I^{t}$ corresponde al valor del test \textit{Welch} para cada uno de los v�xeles de la imagen, lo cual es una medida significativa de la diferencia de medias. De manera intuitiva un alto valor de este elemento indica que hay una diferencia significativa entre las muestras de un espacio y otro, y, por lo tanto, el v�xel en cuesti�n es significativo. 
\par De manera te�rica, altos valores del test (\textit{t-valor}) se corresponden con valores bajos de probabilidad (\textit{p-valor}), donde se referencia por \textit{p-valor} la probabilidad de observar un valor \textit{t-valor}. Queda definida la hip�tesis nula en la igualdad entre las medias de im�genes. Por lo tanto, valores peque�os de $p$ indicar�n el rechazo de la hip�tesis nula en cuesti�n.
\par En nuestro caso, se ha fijado el umbral de decisi�n sobre la hip�tesis nula en $p-valor < 0.05$, esto es, un valor de significancia del 5\%.

\subsection{Segmentaci�n basada en r�giones}
\par Tal y como se ha comentado al principio de este cap�tulo, un aspecto b�sico de el trabajo realizado es la divisi�n o segmentaci�n de las neuroim�genes en r�giones para su estudio posterior, lo cual conlleva un procesado asociado.
\par Dado que se han empleado dos modelos de aprendizaje bien diferenciados ser� necesario llevar a cabo una segmentaci�n de r�giones diferente para cada tipo. Uno de los modelos de aprendizaje est� basada en el estudio de las imagenes 3D de las r�giones mientras que el otro est� basado en caracterizar un vector de v�xeles pertenecientes a la regi�n estudiada.

\par Cabe mencionar los componentes iniciales de este proceso:
\begin{itemize}
\item \textbf{ Vector de v�xeles de  imagen}. Por cada neuroimagen de cada paciente, ya sea una imagen PET o MRI  se tendr� un vector de v�xeles asociado. Este vector contiene los valores de intensidad de los v�xeles dispuesto en forma vectorial en lugar de en una imagen 3D.
\item \textbf{Atlas AAL}.  El Atlas contiene un total de 116 listas distintas, cada una de ellas asociadas a una de las regiones. Cada lista contiene un conjunto de �ndices referidos a la posici�n de los v�xeles que pertenecen a la regi�n en cuesti�n a la que hace referencia la lista. 
\end{itemize}

\subsubsection{Segmentaci�n en vectores 1D}

\par Este tratamiento tiene como objetivo generar un vector de v�xeles para cada una de las 116 r�giones del atlas \textit{AAL}. El procedimiento queda representado en la imagen \ref{fig:Segm1}. El principal aspecto a comentar es que cada regi�n tendr� un n�mero de v�xeles asociado diferente, encarg�ndose el atlas AAL de seleccionar cuales son los v�xeles pertenecientes a cada regi�n tras la previa selecci�n de los voxeles significativos. 
\par Cabe notar como para cada regi�n tendremos un n�mero de voxeles distinto. En la tabla \ref{tb:MRI_voxels_per_region} se observa la amplia diferencia entre r�giones en las im�genes MRI.

\begin{figure}[htp] 

\centering
\includegraphics[scale=0.4]{images/Segmentacion_1D.png}
\caption{Proceso de segmentaci�n de vectores por regi�n}
\label{fig:Segm1}
\end{figure} 

\begin{table}[b]
\centering
\begin{tabular}{ c | c } 
Regi�n &  N�V�xeles \\ \hline
1 & 8272 \\ \hline
20 & 5535 \\ \hline
40 & 2708 \\ \hline
60 & 5191 \\ \hline
80 & 567 \\ \hline
100 & 4260 \\ \hline
116 & 560 
\end{tabular}
\caption{N�mero de v�xeles en imagen MRI por cada regi�n}
\label{tb:MRI_voxels_per_region}
\end{table}




\subsubsection{Segmentaci�n en im�genes 3D}

\par En el caso de la obtenci�n de las r�giones cerebrales en imagenes 3D se ha realizado un procesado basado en la obtenci�n de una m�scara 3D sobre los �ndices de v�xeles del atlas AAL. Para ello, se ha reconstruido el atlas, extrayendo de aqu� los l�mites en cada una de las dimensiones de la posici�n de la regi�n evaluada. 
\par Posteriormente, se ha usado este conocimiento de la posici�n exacta de los v�xeles en 3D para llevar a cabao la extracci�n de la regi�n en cuesti�n. Este proceso ha sido esquematizado en la imagen \ref{fig:Segm3}. En la imagen \ref{fig:Segm3_example} se pueden observar dos r�giones extra�das y representadas en 3D. 

\begin{figure}[htp] 
\centering
\includegraphics[scale=0.35]{images/Segmentacion_3D.png}
\caption{Proceso de segmentaci�n de vectores por regi�n}
\label{fig:Segm3}
\end{figure} 

\begin{figure}[htp] 
\centering
\includegraphics[scale=0.5]{images/pet_3d.png}
\caption{Ejemplo de R�giones de im�genes PET segmentadas.(Izquierda) Regi�n N� 20. (Derecha) Regi�n N� 30. Capturas de im�genes 3D tomadas con el programa \textit{MRIcrGL}}
\label{fig:Segm3_example}
\end{figure} 

\subsection{Reconstrucci�n a partir de R�giones}


\newpage
\section{Modelos Generativos}
asddfasdf
\newpage
\section{Modelo de Clasificaci�n}
\par El dise�o y aplicaci�n de un modelo de clasificaci�n es de especial inter�s dado que la capacidad de clasificaci�n esta intimamente relacionada con el m�todo de extracci�n caracter�sticas empleado, que en nuestro caso ser� el autoencoder variacional.


\par Tal y como ya se ha explicado pr�viamente, el autoencoder es capaz generar un conjunto de caracter�sticas representativas en lo que se denomina espacio latente, siendo estas variables obtenidas sobre las que posteriormente se les aplicar� el proceso de clasificaci�n. Estas variables latentes ser�n genereadas por regi�n, por lo que al final tendremos tantos vectores de variables latentes como n�mero de r�giones se est�n evaluando.

\par Es importante notar que el procesamiento de las r�giones por separado implica la aplicaci�n de un procedimiento de clasificaci�n en el cu�l debemos de unir la informaci�n de evaluaci�n de cada regi�n. En la figura \ref{fig:main_clasificacion} se puede apreciar lo mencionado anteriormente.

\par A lo largo de esta secci�n se detallar� el m�todo explicando el procedimiento realizado y mencionando algunas de las problem�ticas encontradas como son la extracci�n de caracter�sticas por regi�n o el mezclado de informaci�n para los dos tipo de im�genes MRI.


\begin{figure}[htp] 
\centering
\includegraphics[scale=0.5]{diagrams/ClasificacionMainEsquema.png}
\caption{Esquema b�sico de Autoencoder. Cabe notar como ser� el c�digo generado en la capa latente lo que se emplear� para la clasificaci�n posterior}
\label{fig:main_clasificacion}
\end{figure} 

\par 


\subsection{Extracci�n de Caracter�sticas por Regi�n}
\par El proceso de extracci�n de caracter�sticas aplica el Autoencoder Variacional. Una esquematizaci�n de dicho proceso se puede observar en la imagen \ref{fig:Extracci�n_1}. Es de esperar que el aumento del tama�o de la capa latente, esto es, que haya m�s neuronas en dicha capa conlleve una mejora en los resultados de la clasificaci�n dado que se ha comprimido menos la informaci�n de las im�genes.

\begin{figure}[htp] 
\centering
\includegraphics[scale=0.5]{diagrams/RegionExtractFeatures.png}
\caption{Diagrama b�sico del proceso de extracci�n de caracter�sticas aplicando el Autoencoder Variacional}
\label{fig:Extracci�n_1}
\end{figure} 



\par Se pueden diferencia dos fases, la de entrenamiento y la de extracci�n de caracter�sticas, mientras que el tercer bloque de la figura \ref{fig:Extracci�n_1} hace referencia al vector de variables esperadas a la salida.
\begin{itemize}
\item \textbf{Entrenamiento del Autoencoder}. El entrenamiento tiene como objetivo caracterizar el Autoencoder en funci�n del tipo de muestra, en nuestro caso en funci�n de la tipolog�a de neuroimagen de cada una de las r�giones evaluadas. Es este entrenamiento el proceso mas costoso, tanto computacionalmente como en t�rminos de coste en el desarrollo de este modelo de clasificaci�n. 
\par Este proceso permite la extracci�n efectiva de caracter�sticas ya que es el que ajusta el conjunto de par�metros encargados de generar el conjunto de variables latentes en funci�n del tipo de imagen de entrada. 

\item \textbf{Codificaci�n}. Dado el conjunto de v�xeles de cada neuroimagensociado a un tipo de regi�n, este proceso se encargar� de generar el conjunto de varibles latentes.

\end{itemize}

\par En esta fase es posible aplicar tanto el VAE de capas densas como el CVAE dado que lo importante es garantizar que la salida ser� un vector de variables sin importar el procedimiento de extracci�n empleado. No obstante, el tipo de VAE empleado modificar� el procesado previo de las imagenes dado que si se emplea el VAE de redes densa se deber�n vectorizar las im�genes mientras que si es el CVAE se deber� delimitar en 3D las distintas regiones.

\par Un aspecto diferencial en el tratamiento de las im�genes PET y MRI es que para las MRI tenemos dos modalidades de im�genes, las de materia blanca y las de materia gris, es por ello que para las im�genes MRI necesitaremos dos \textit{Autoencoders} lo cual duplica el coste computacional. En la figura \ref{fig:Extracion_MRI} queda representado este concepto

\begin{figure}[htp] 
\centering
\includegraphics[scale=0.5]{diagrams/RegionExtractFeaturesMRI.png}
\caption{Diagrama b�sico del proceso de extracci�n de caracter�sticas para las im�genes MRI}
\label{fig:Extracci�n_MRI}
\end{figure} 

\par Por lo tanto, para el caso de las im�genes MRI se tendr� a la salida un vector de caracter�sticas que ser� la concatenaci�n de las variables latentes obtenidas a partir de las im�genes de materia blanca y materia gris.





\newpage
\subsection{Evaluaci�n por Regi�n}

\


\begin{figure}[htp] 
\centering
\includegraphics[scale=0.5]{diagrams/ExtracccionCaracteristicasVAE.png}
\caption{Esquema del proceso de extracci�n de caracter�sticas aplicando el autoencoder variacional. Es importante notar como el procesamiento y al extracci�n de las caracter�sticas para cada una de las r�giones evaluadas se reliaza por separado y es posteriormente cuando la informaci�n de las distintas r�giones es concatenada}
\label{fig:Clasificacion_2}
\end{figure} 



\newpage
\subsection{Evaluaci�n Conjunta}

\begin{figure}[htp] 
\centering
\includegraphics[scale=0.5]{diagrams/Autoencoder.png}
\caption{Esquema b�sico del proceso de clasificaci�n}
\label{fig:Codifiaci�n}
\end{figure} 

\par 



\begin{figure}[htp] 
\centering
\includegraphics[scale=0.5]{diagrams/Clasificacion.png}
\caption{}
Falta el SVM 
\label{fig:Clasificacion_1}
\end{figure} 






\begin{figure}[htp] 
\centering
\includegraphics[scale=0.5]{diagrams/MRIClasificacionVAE.png}
\caption{Este esq}
\label{fig:Clasificacion_3}
\end{figure} 


\par En el caso de las im�genes MRI el procedimiento empleado en el procesamiento de las r�giones es parcialmente diferente. 



Falta el SVM 




































\chapterend{}
%%%%%%%%%%%%%%%%%%%%%%%%%%%%%%%%%%%%%%%%%%%%%%%%%%%%%%%%%%%%%%%%%%%
%%% Documento LaTeX 																						%%%
%%%%%%%%%%%%%%%%%%%%%%%%%%%%%%%%%%%%%%%%%%%%%%%%%%%%%%%%%%%%%%%%%%%
% T�tulo:		Cap�tulo 2
% Autor:  	Ignacio Moreno Doblas
% Fecha:  	2014-02-01
% Versi�n:	0.5.0
%%%%%%%%%%%%%%%%%%%%%%%%%%%%%%%%%%%%%%%%%%%%%%%%%%%%%%%%%%%%%%%%%%%
\chapterbegin{Resultados}
\label{chp:Resultados}


\section{Clasificaci�n}

\subsubsection{M�tricas de Evaluaci�n}

\par Hay diferentes m�tricas de evaluaci�n referidas a las caracter�stiscas del diagn�stico. Algunas se utilizan para evaluar la capacidad discriminativa de la prueba y otras para estimar su propiedad de predicci�n, siendo esta �ltima muy sensible a las caracter�sticas de las poblaciones evaluadas. 
\par Con objeto de valorar la calidad del test de diagn�stico es necesario saber cu�n buena y confiable es una prueba. Esta cuantificaci�n es llevada por el conjutno de medidas que se explicar�n a continuaci�n, las cu�les est�n basadas en los �ndices expuestos en la tabla \ref{table:metrics}.


\begin{table}[]
\centering
\label{table:metrics}
\begin{tabular}{C{3cm}C{3cm}|C{3cm}|C{3cm}|}
\cline{3-4}
  &  & \multicolumn{2}{c|}{	\textbf{Condici�n Real}} \\ 
\cline{3-4} 
  &  & \textbf{Positivo} & \textbf{Negativo} \\ 
\hline
\multicolumn{1}{|c|}{\multirow{2}{*}{\textbf{Predicci�n}}} & \textbf{Positivo} &  Verdadero Positivo (VP)                 &               Falso    Positivo (FP) \\ \cline{2-4} 
\multicolumn{1}{|c|}{}                            & \textbf{Negativo}     &  Falso Negativo (FN)               & Verdadero Negativo (VN)              \\ \hline
\end{tabular}
\caption{Tabla de Nomenclatura Estad�stica en Clasificaci�n}
\end{table}

\subsubsection{Medida F}

\begin{center}
\begin{equation} \label{eq:p_x_gausiana}
Valor-F = \frac{*VP}{2*VP + FP + FN}
\end{equation}
\end{center}


\subsubsection{Recall. Sensibilidad}
\par La precisi�n mide la capacidad de una prueba diagn�stica de identificar los sujetos enfermos con respecto al total de sujetos enfermos de la poblaci�n.

\begin{center}
\begin{equation} \label{eq:p_x_gausiana}
Precision = \frac{VP}{VP + FN}
\end{equation}
\end{center}


\subsubsection{Especificidad. Precisi�n}
\par La especificidad hace referencia a la capaciad de la prueba a identificar a los sujetos enfermos y excluir a aquellos que  no lo est�n. 
\begin{center}
\begin{equation} \label{eq:p_x_gausiana}
Precision = \frac{VP}{VP + FP}
\end{equation}
\end{center}

\subsubsection{Precision. Accuracy}
\par Indica la capacidad la del m�todo de clasificaci�n de identificar tanto pacientes sanos como pacientes enfermos con respecto al total de la poblaci�n.
\begin{center}
\begin{equation} \label{eq:p_x_gausiana}
F1 = \frac{VP + VN}{N�Samples}
\end{equation}
\end{center}

\subsubsection{Curva ROC}
La curva ROC (del ingl�s \textit{Receiver Operating Characteristic}) es una t�cnica gr�fica que
nos permite evaluar la precisi�n del modelo estad�stico para clasificar dos clases, AD y
NOR. La curva se obtiene calculando la sensibilidad (proporci�n de resultados positivos
verdaderos) y la especificidad del modelo en cada punto de corte posible, y trazando la
sensibilidad frente a 1-especificidad (proporci�n de resultados falsos positivos).

\par Cada punto en el espacio ROC muestra el equilibrio entre la sensibilidad y la
especificidad, es decir, que el aumento de sensibilidad va acompa�ado de una disminuci�n en la especificidad. 

\begin{figure}[!hb]
\centering
\includegraphics[scale=0.5]{images/curva_roc.png}
\caption{Representaci�nd de la curva ROC}
\label{fig:curva_roc}
\end{figure}


\par Cada punto de corte de una prueba de diagn�stico define un �nico punto en el
espacio ROC, los diferentes puntos posibles definen la curva ROC. Esto es an�logo a lo dicho para un solo punto, por tanto cuanto m�s se acerquen los puntos de la curva ROC a la coordenada ideal, m�s exacta ser� la prueba y viceversa, cuanto m�s se aleje peores
resultados obtendremos.
\subsubsection{�rea bajo la curva}

\par El �rea bajo la curva ROC se denomina AUC (del ingl�s \textit{area under the curve}) y se interpreta como el promedio de precisiones positivas y negativas. Este �ndice es especialmente �til en los estudio
comparativo de pruebas de diagn�stico. Siendo deseable comparar toda la curva ROC
en lugar de en un punto particular. 

\subsection{Resultados Clasificaci�n}


\begin{figure}[!hb]
\centering
\includegraphics[scale=0.45]{images/SweepLatentLayerPET.png}
\caption{Representaci�nd de la curva ROC}
\label{fig:curva_roc}
\end{figure}

%\minitoc
\chapterend{}
\chapterbeginx{Conclusiones y l�neas futuras}

\par En este trabajo se ha analizado la t�cnica generativa denominada Autoencoder Variacional realiz�ndose dos implementaciones Software de dicha t�cnica utilizando el lenguaje \textit{Python}. Esta modelo de aprendizaje no supervisado est� basado en aprendizaje profundo, es por ello que se ha empleado la librer�a \textit{Tensorflow} que da soporte a la implementaci�n de redes neuronales para su ejecuci�n tanto en CPU como en GPU. Algunos de los aspectos a destacar, as� como algunas de las contribuciones realizadas durante el desarrollo del trabajo son las siguientes:

\begin{itemize}

\item Se ha implementado un modelo de VAE basado en redes neuronales densas. El modelo es facilmente configurable tanto en variables de configuraci�n como es la tasa de aprendizaje o en par�metros estructurales como son el n�mero de capas del modelo o el n�mero de neuronas para cada capa.

\item Se ha implementado un modelo de VAE basado en redes neuronales convolucionales. En este caso las variables de configuraci�n son modificables f�cilmente, y se han habilitado una serie de modelos que habilitan la posibilidad de selecionar entres distintas estructuras para el sistema.

\par Esta capacidad  para variar  tanto los par�metros del modelo como ciertos aspectos estructurales permite evaluar difentes configuraciones con objeto de obtener la �ptima. 

\item Se ha evaluado la capacidad de los modelos implementados de discernir entre pacientes AD y NOR emplenado tanto muestras PET como MRI. 
\par Para las imagenenes MRI se ha obtenido una precisi�n  m�xima de clasificaci�n de hasta el 84 \% para el VAE. En otros estudios, emple�ndose este mismo espacio muestral se han conseguido resultados superiores al 90 \%, por lo que son bastante superiores a los conseguidos en este trabajo. No obstante el objetivo no era conseguir un clasificador �ptimo que consiguiera competir con los modelos actuales sino demostrar la capacidad de discernir entre las neuroimagenes de un sujeto NOR y las de uno AD.
\par Para las im�genes PET se ha obtenido una precisi�n m�xima en torno al 90 \%. De nuevo estos resultados no compiten con los modelos del estado del arte actual que ronda el 95 \% con este mismo espacio muestral.

\item El objetivo principal del trabajo era conseguir realizar la s�ntesis de neuroim�genes 3D a partir de la caracterizaci�n del VAE. No obstante, las s�ntesis realizadas utilzando el c�digo generado por las propias im�genes del espacio muestral han generado unas im�genes que apenas se asemejaban a las originales. 
\par Tanto para el modelo VAE de redes densas como para el CVAE se ha tenido un resultado nefasto para la s�ntesis aunque hay diversos aspectos en la implementaci�n de ambos modelos que son diferentes. Esto genera la duda de si el problema est� en la implementaci�n del modelo o en el m�todo VAE en s� que no se ajuste al objetivo de este trabajo. 
\par En las simulaciones de visualizaci�n del c�digo latente se ha comprobado como es posible separar para algunas regiones el c�digo generado para im�genes de sujetos AD y para sujetos NOR. Esto nos indica que el autoencoder es capaz de generar un c�digo diferente para los dos tipos de sujetos. Sin embargo durante la fase de reconstrucci�n se tiene que los c�digos generan de nuevo el mismo tipo de imagen. Una imagen final que puede representar un valor medio de imagen que permite reducir el error de regeneraci�n durante el proceso de entrenamiento.
\par En resumen, es realmente desconcertante que c�digos latentes diferenciables entre NOR y AD generen finalmente un mismo tipo de imagen, o al menos muy similar. 

\item Otro aspecto importante es que el uso de redes convolucionales 3D como modelo extracci�n de caracter�sticas para el VAE no es una metodolog�a que se haya usado en muchos trabajos relativos a esta tem�tica, al menos durante la fase inicial de este proyecto no se encontro ning�n c�digo de implementaci�n sobre \textit{Tensorflow} de este tipo de red. Adem�s, \textit{Tensorflow} no dispone de la funci�n encargada de realizar el proceso de agrupamiento (\textit{pooling}) para im�genes 3D, lo cu�l puede conlleva la imposibilidad de probar esta funcionalidad de reducci�n de caracter�sticas en el sistema. 

\subsection{L�neas futuras}
\par En este trabaja la implementaci�n se ha apoyado en \textit{Tensorflow} para la realizaci�n de los dise�os. Existen diversas librer�as sobre \textit{Python} que facilitan el mismo conjunto de funcionalidades que \textit{Tensorflow}, tales librer�as son \textit{Keras} o \textit{Theano}, es por ello que una posible l�nea de trabajo ser�a realizar la implementaci�n sobre alguna de estas librer�a y evaluar si los resultados de la s�ntesis de im�genes son similares. Esto nos permitir�a descartar si los problemas de este trabajo son debidos a las implementaciones aqu� realizadas.

\par Adem�s de VAE, existen otros tipos de t�cnicas generativas que se basan en el uso de aprendizaje profundo. Las redes generativas adversarias, que fueren brevemente comentadas en la secci�n \ref{sec_redesAdversarias}, constituyen un m�todo caracterizado por generar una s�ntesis de im�genes m�s realista que el VAE pero que es ciertamente m�s complejo de entrenar y de llegar a una configuraci�n �ptima de los par�metros. 
\end{itemize}

\chapterend



%\input{D2.AppendixB.tex}

%\input{D3.AppendixC.tex}

% Formato de documento en la parte final.
\backmatter
%Hace que los cap�tulos y t�tulos nivel inferior no aparezcan numerados (lo que es ideal para conclusiones o notas finales).

% Bibliograf�a
%\input{E1.Bibliografia.tex}
%\begin{thebibliography}{9}

\bibitem{Alzheimer1} 
	Henley, David B.,
	Sundell, Karen L.,
	Sethuraman, Gopalan,
 	Siemers, Eric R.,
	2011,	
	\textit{Safety profile of Alzheimer's disease populations in Alzheimer's Disease Neuroimaging Initiative and other 18-month studies},
 	407-416


\bibitem{ReservaCognitiva}
	Rodríguez Álvarez M., Sánchez J. L.,
	2004,
	\textit{Reserva cognitiva y demencia},
	2004, vol. 20, nº 2 (diciembre), 175-186 
	
\bibitem{DiagnosticoNormal}
 Petrella J.R, Coleman R.E., Doraiswamy P.M., Neuroimaging and Early Diagnosis
of Alzheimer Disease: A Look to the Future. Radiology 2003; 226:315?336.



\bibitem{MRIsurvey}
	Rémi Cuingnet et al, 
	\textit{Automatic classification of patients with Alzheimer's disease from
	structural MRI: A comparison of ten methods using the ADNI database},
	 Neuroimage,
	vol. 56, no. 2, pp. 766-781, 2011.

  \end{thebibliography}


\begin{thebibliography}{9}

\bibitem{Alzheimer1} 
	Henley, David B.,
	Sundell, Karen L.,
	Sethuraman, Gopalan,
 	Siemers, Eric R.,
	2011,	
	\textit{Safety profile of Alzheimer's disease populations in Alzheimer's Disease Neuroimaging 			Initiative and other 18-month studies},
 	407-416

\bibitem{ReservaCognitiva}
	Rodr�guez �lvarez M., S�nchez J. L.,
	2004,
	\textit{Reserva cognitiva y demencia},
	2004, vol. 20, n� 2 (diciembre), 175-186 
	
\bibitem{DiagnosticoNormal}
 Petrella J.R, Coleman R.E., Doraiswamy P.M., Neuroimaging and Early Diagnosis
of Alzheimer Disease: A Look to the Future. Radiology 2003; 226:315?336.

\bibitem{MRIsurvey}
	R�mi Cuingnet et al, 
	\textit{Automatic classification of patients with Alzheimer's disease from
	structural MRI: A comparison of ten methods using the ADNI database},
	 Neuroimage,
	vol. 56, no. 2, pp. 766-781, 2011.

\bibitem{introPET}
	W. Cai, 
	D. Feng, 
	R. Fulton, 
	\textit{Content-based retrieval of dynamic PET functional images} 
	IEEE Trans. Inf. Technol. Biomed., vol. 4, no. 2, pp. 152-158, 2000.

\bibitem{PETMRIMultimodal}
	Ortiz, A.; 
	Fajardo, D.; 
	G�rriz, J.M.; 
	Ram�rez, J.; 
	Mart�nez-Murcia, F.J.,
	\textit{Multimodal
image data fusion for Alzheimer's Disease diagnosis by sparse representation}, 
International Conference on Innnovation in Medicine and Healthcare (InMed), 2014


\bibitem{PETMRIMultimodal2}
	Daoqiang Zhanga, 
	Yaping Wanga, 
	Luping Zhoua, 
	Hong Yuana, 
	Dinggang Shena,
	\textit{Multimodal Classification of Alzheimer's Disease and Mild Cognitive Impairment}
	Neuroimage. 2011 April 1; 55(3): 856-867


\bibitem{SVMtrees}
	D. Salas-Gonzalez, 
	J. M. Gorriz, 
	J. Ram�rez, 
	M. Lopez, 
	I Alvarez,
	\textit{Compute-aided diagnosis of Alzheimer's disease using support vector machines and classification trees}
	Phys. Med. Biol. 55 (2010) 2807-2817


\bibitem{Survey}
	Ruaa Adeeb Abdulmunem Al-falluji
	\textit{MRI based Techniques for Detection of Alzheimer: A Survey}
	International Journal of Computer Applications (0975 - 8887)
	Volume 159 - No 5, February 2017

\bibitem{Survey2}
	S.Mareeswari1, 
	Dr.G.Wiselin,
	\textit{A survey Early Detection of Alzheimer's Disease using different techniques}
	International Journal on Computational Science and Applications (IJCSA) Vol.5, No.1,February 2015

\bibitem{CoD}
Bellman RE, 
1961,
\textit{Adaptive control processes: a guided tour}, Princeton
 University Press.


\bibitem{CoD2}
	David L. Donoho
	Department of Statistics
	\textit{High-Dimensional Data Analysis: The Curses and Blessings of Dimensionality}
	August 8, 2000

\bibitem{ReduceDimensionality}
	Benson Mwangi, 
	Tian Siva Tian,  
	Jair C. Soares
	\textit{A review of feature reduction techniques in neuroimaging}
	Neuroinformatics. 2014 April ; 12(2): 229-244. doi:10.1007/s12021-013-9204-3.

\bibitem{DeepLearning1}
	Siqi Liu, 
	Sidong Liu,
	\textit{EARLY DIAGNOSIS OF ALZHEIMER'S DISEASE WITH DEEP LEARNING}

\bibitem{DeepLearning2}
	Saman S., 
	Ghassem T.,
	\textit{Classification of Alzheimer's Disease Structural MRI Data by Deep Learning Convolutional Neural Networks}
	22 Jul 2016

\bibitem{AutoEncoderVariational}
	Diederik P. Kingma,
	Max Welling,
	\textit{Auto-Encoding Variational Bayes}
	1 May 2014

\bibitem{AutoEncoderOrigin1}
	 D. E. Rumelhart, G. E. Hinto, and R. J. Williams. 
	 Learning Internal Representations by Error Propagation
	 9 October 1986

\bibitem{AutoEncoderSurvey}
	: I. Guyon, 
	G. Dror, 
	V. Lemaire, 
	G. Taylor and 
	D. Silver
	Autoencoders, Unsupervised Learning, and Deep Architectures
	2012

\bibitem{AutoEncoderDenoising}
	P. Vincent
	H. Larochelle
	I. Lajoie
	Stacked Denoising Autoencoders: Learning Useful Representations in a Deep Network with a Local Denoising Criterion
	2010

\bibitem{DeepGenerativeModels}
	Danilo J. Rezende, 
	Shakir Mohamed, 
	Daan Wierstra
	Stochastic Backpropagation and Approximate Inference in Deep Generative Models

\bibitem{AD_historic}
	Bennett, D. A.,
	Evans, D. A., 
	1922.
	Alzheimer's Disease.
	Disease-a-Month 38(1), 7-64.

\bibitem{ADhistoric2}
	Nakako, S.,
	Kato, T.,
	Nakamura.,
	1996.
	Acetylcholinesterase activity in cerebrospinal fluid of patients with alzhimer's disease and senile dementia.
	Journal of the Neurological Sciences
	75(2).
\bibitem{mitocondria2}
	Rodriguez-Violante M.,
	Cervantes A.,
	Vargas S.,
	2010.
	Papel de la funci�n mitocondrial en las enfermedades neurodegenerativas.
	Arch Neurocien (Mex)  Vol. 15, N�1: 39-46

\bibitem{mitocondriaGeneral}
	Swerdlow, R.,
	2011
	Brain agin, alzheimer's diseas, and mitochondira.
	Biochim Biophys Acta 1812(12), 1630-1639


\bibitem{Nations}
	Nations U.,
	2008
	Department of economic and social affairs, world population prospects. 

\bibitem{Predemencia1}
	 Arn�iz E,., 
	 Almkvist O., 
	 2003,. 
	 Neuropsychological features of mild cognitive impairment and preclinical Alzheimer's disease. 



\bibitem{Predemencia2}
 	Palmer K., 
 	Berger A. K., 
 	Monastero R., 
 	Winblad B., 
 	B�ckman L., 
 	Fratiglioni L. 
 	2007. 
 	Predictors of progression from mild cognitive impairment to Alzheimer disease. 
 	Neurology 68 (19): 1596-1602. 
 	PMID 17485646. doi:10.1212/01.wnl.0000260968.92345.3f.

\bibitem{DemenciaMemoria}
	 Carlesimo GA, Oscar-Berman M (junio de 1992). �Memory deficits in Alzheimer's patients: a comprehensive review�. Neuropsychol Rev 3 (2): 119-69. PMID 1300219.

 \bibitem{DemenciaComunicacion}
 	 Frank EM (septiembre de 1994). �Effect of Alzheimer's disease on communication function�. J S C Med Assoc 90 (9): 417-23. PMID 7967534.
\bibitem{ADconducta}
	Volicer L, 
	Harper DG, 
	Manning BC, 
	Goldstein R, 
	Satlin A., 
	Mayo de 2001. 
	Sundowning and circadian rhythms in Alzheimer's disease�. Am J Psychiatry 158 (5): 704-711. PMID 11329390. 


\bibitem{hipocampo}
	Mu Y., 
	Gage FH,
	2011 Dec,
Mol Neurodegener. 2011 Dec 22;6:85. doi: 10.1186/1750-1326-6-8
Adult hippocampal neurogenesis and its role in Alzheimer's disease.


\bibitem{atlasAlzheimer}
	FeldMan H. H.,
	Atlas of Alzheimer's Disease.
	Informa Healthcare

\bibitem{MainADNI}
	Susanne G. Mueller, 
	Michael W. Weiner, 
	Neuroimaging Clin N Am. 2005 November ; 15(4): 869?xii.
	The Alzheimer?s Disease Neuroimaging Initiative
	
\bibitem{}
 Gorji, H. T.; Haddadnia, J. (2015-10-01). "A novel method for early diagnosis of Alzheimer's disease based on pseudo Zernike moment from structural MRI". Neuroscience. 305: 361?371. ISSN 1873-7544. PMID 26265552. doi:10.1016/j.neuroscience.2015.08.013.

\bibitem{}
 Zhu, Xiaofeng; Suk, Heung-Il; Shen, Dinggang (2014-10-15). "A novel matrix-similarity based loss function for joint regression and classification in AD diagnosis". NeuroImage. 100: 91?105. ISSN 1095-9572. PMC 4138265?Freely 
 accessible. PMID 24911377. doi:10.1016/j.neuroimage.2014.05.078.

\bibitem{MRIOrigin3}
Wright G., 
Magnetic Resonance Imaging
EEE Signal Process. Mag. 14 (1997)56?66.


\bibitem{MRIOrigin}
Lauterbur P. C., 1973. 
"Image Formation by Induced Local Interactions: Examples of Employing Nuclear Magnetic Resonance". 
Nature. 242 (5394): 

\bibitem{MRIOrigin2}
 "Britain's brains produce first NMR scans". 
 New Scientist: 588. 1978.

\bibitem{MRINoise}
Snehal More, 
V.V.Hanchate,
"A Survey on Magnetic Resonance Image Denoising Methods"
International Research Journal of Engineering and Technology (IRJET),
Volume: 03 Issue: 05 | May-2016 

\bibitem{rbm}
Asja Fischer,
Christian Igel,
\textit{An Introduction to Restricted Boltzmann Machines}


\bibitem{denoising_autoencoder}
Vincent H.,
Larochelle H.,
Lajoie. I.,
\textit{Stacked Denoising Autoencoders: Learning Useful Representations in
a Deep Network with a Local Denoising Criterion}

\bibitem{denoising_autoencoder2}
Vincent, Pascal
Larochelle, Hugo
Bengio, Yoshua
\textit{Extracting and Composing Robust Features with Denoising Autoencoders},
  (ICML'08), pages 1096 - 1103, 2008

\bibitem{gan}
 Goodfellow, Ian J.; 
 Pouget-Abadie, Jean; 
 Mirza, Mehdi; 
 Xu, Bing; 
 Warde-Farley, 
 David; Ozair, 
 Sherjil; Courville, 
 Aaron; Bengio, Yoshua 
 (2014), 
 "Generative Adversarial Networks". arXiv:1406.2661?

\bibitem{gan_example}
Salimans T., 
Goodfellow I., 
Zaremba., 
Cheung, Vicki; Radford, Alec; Chen, Xi (2016).
"Improved Techniques for Training GANs". arXiv:1606.03498?



\bibitem{vae_2d}
Eunbyung Park 
University of North Carolina at Chapel Hill


\bibitem{mri_survey}
\textit{MRI based Techniques for Detection of Alzheimer: A
Survey} 
Ruaa Adeeb Abdulmunem Al-falluji
University of Babylon,
International Journal of Computer Applications (0975 ? 8887),
Volume 159 ? No 5, February 2017
\bibitem{ica}

Yang, Wenlu, et al. "Independent component analysis-
based classification of Alzheimer's disease MRI
data." Journal of Alzheimer's disease 24.4 (2011): 775-
783.

\bibitem{pca}
18F-FDG PET imaging analysis for computer aided
Alzheimer?s diagnosis? I.A. Ill�n, J.M. G�rriz,J.
Ram�rez, D. Salas-Gonzalez, M.M. L�pez, F. Segovia,
R. Chaves, M. G�mez-Rio c, C.G. Puntonet,
Information Sciences 181 903?916, 2011.

\bibitem{wavelet}

Herrera, Luis Javier, et al. "Classification of MRI
Images for Alzheimer's Disease Detection." Social
Computing
(SocialCom),
2013
International
Conference on. IEEE, 2013.


\bibitem{knn}

Zhang, H., Berg, A.C., Maire, M., Svm-knn, J.M.:
Discriminative nearest neighbor classification for visual
category recognition. In: CVPR ?06, pp. 2126?2136.
IEEE Computer Society, Los Alamitos, CA, USA
(2006)

\bibitem{deep_learning}
Weiner, Michael (2017). "Recent publications from the Alzheimer's disease neuroimaging initiative: reviewing progress toward improved AD clinical trials.". Alzheimer's and dementia.


\bibitem{deep_learning_2}
 Suk, Heung-Il; Lee, Seong-Whan; Shen, Dinggang; Alzheimer?s Disease Neuroimaging Initiative (2016-06-01). 
 "Deep sparse multi-task learning for feature selection in Alzheimer's disease diagnosis". 

\bibitem{deep_learning_3}

 Zu, Chen; Jie, Biao; Liu, Mingxia; Chen, Songcan; Shen, Dinggang; Zhang, Daoqiang; Alzheimer?s Disease Neuroimaging Initiative (2016-12-01). "Label-aligned multi-task feature learning for multimodal classification of Alzheimer's disease and mild cognitive impairment". Brain Imaging and Behavior.

\bibitem{cnn_ad_1}
Hosseini-Asl E., 
Keynton R., 
El-Baz A.,
\textit{Alzheimer?s disease diagnostics by adaptation of 3d convolutional network}
(Julio 2016), 
arXiv:1607


\bibitem{cnn_ad_2}
Suk H-I., 
Shen D., 
\textit{Deep Learning-Based Feature Representation for AD/MCI Classification}, 
Medical image computing and computer-assisted intervention?(MICCAI), International Conference on Medical Image Computing and Computer-Assisted Intervention. 2013;16(0 2):583-590.

\bibitem{tensorflow}
Abadi M, 
Agarwal A., 
Barham P., 
B. Eugene,. 
\textit{TensorFlow:
Large-Scale Machine Learning on Heterogeneous Distributed Systems}, 
November 9, 2015

\bibitem{DNN_1}
D.E. Rumelhart, J.L. MacClelland (eds.) (1986). Parallel Distributed Processing.
Vol 1. Foundations, MIT Press

\bibitem{FA_1}
Glorot, X.,
Bengio, Y.,
(2010).
\textit{Understanding the difficulty of training deep feedforward neural networks}
Journal of Machine Learning Research - Proceedings Track. 9. 249-256.  

\bibitem{FA_2}
LeCun, Yann; Bengio, Yoshua; Hinton, Geoffrey (2015). "Deep learning". Nature. 521 (7553): 436?444. Bibcode:2015Natur.521..436L. PMID 26017442. doi:10.1038/nature14539.

\bibitem{DNN_training}
Xavier Glorot and Yoshua Bengio,
\textit{Understanding the difficulty of training deep feedforward neural networks},
(2010)
In Proceedings of the International Conference on Artificial Intelligence and Statistics (AISTATS?10). Society for Artificial Intelligence and Statistics

\bibitem{CNN}
LeCun,  Y.,  Boser,  B.,  Denker,  J.  S.,
Backpropagation  applied  to  handwritten  zip  code
recognition.
Neural Comput.
, 1(4):541?551, 1989

\bibitem{CNN_1}
Visualizing and Understanding Convolutional Networks
Ciresan, D. C., Meier, J., and Schmidhuber, J.  Multi-
column  deep  neural  networks  for  image  classifica-
tion.  In
CVPR
, 2012.

\bibitem{CNN_2}
Krizhevsky,  A.,  Sutskever,  I.,  and  Hinton,  G.E.   Im-
agenet classification with deep convolutional neural
networks.  In
NIPS
, 2012.

\bibitem{SVM_1}
Y. Zhang and W. Zhou, Multifractal analysis and
relevance vector machine-based automatic seizure
detection in intracranial, Int. J. Neural Syst. 25(6)
(2015) 1?14.

\bibitem{SVM_2}
E. Castillo, D. Peteiro-Barral, B. Guijarro Berdinas and O. Fontenla-Romero, Distributed one-class support vector machine, Int. J. Neural Syst. 25(7)
(2015) 1?17

\bibitem{SVM_3}
A. Hidalgo-Mun? noz, J. G�rriz, J. Ram�rez and
P. Padilla, Regions of interest computed by svm
wrapped method for Alzheimers disease examination
from segmented mri, Front. Aging Neurosci. 6(20)
(2014)

\bibitem{ValidacionCruzada}
Hatie T., Tibshirani R., Friedman J., The elements of statistical learning, data mining, inferencen, and prediction. Springer, 2008

\bibitem{AAL}
G. Flandin, F. Kherif, X. Pennec, D. Riviere,
N. Ayache and J.-B. Poline, fMRI data analysis, Proceedings of the IEEE Int. Symposium on Biomedical
Imaging (7?11 July 2002, Prague, Czech Republic),
pp. 907?910.

\bibitem{VBM}
Ashburner J., 
Friston KJ.
\textit{Voxel-based morphometry. The methods}
Neuroimage. 2000 Jun;11(6 Pt 1):805-21.

\bibitem{VBM_2}
J. Ashburner and T. Group, SPM8 Manuel, Vol. 12
(Functional Imaging Laboratory, Institute of Neurology, UK, 2011).


\bibitem{VBM_3}
Structural Brain Mapping Group, Department
of Psychiatry, Available at http://dbm.neuro.unijena.de/vbm8/VBM8-Manual.pdf [Accessed on 10
March 2014]

\bibitem{PET_norm}
N. S. V. Villemagne, 
S. Berlangieri, 
S. Lee, M. Cherk,
S. Gong, 
U. Ackermann, 
T. Saunder, 
H. TochonDanguy, G. Jones, C. Smith, G. OKeefe, C. Masters and C. Rowe, 
\textit{Visual assessment versus quantitative assessment of 11c-pib pet and 18f-fdg pet for
detection of Alzheimer?s disease}, J. Nucl. Med. 48(4)
(2004) 34?41.

\bibitem{Measures_1}
Wen Zhu, Nancy Zeng, Ning Wang. 
\textit{Sensitivity, Specificity, Accuracy, Associated
Confidence Interval and ROC Analysis with Practical SAS Implementations} 
1K\&L consulting services, Inc, Fort Washington, PA Octagon Research Solutions, Wayne, PA,
2010.

\bibitem{Measures_2}
Hwee Bee Wong, Gek Hsiang Lim. 
\textit{Measures of Diagnostic Accuracy: Sensitivity, Specificity, PPV and NPV} 
Health Services Research and Evaluation Division,
Ministry of Health, Singapore


\bibitem{learningRate}
Mandic, Danilo P. and Chambers, Jonathon A.
\textit{Towards the Optimal Learning Rate for Backpropagation}
2000,
01 Feb,
Neural Processing Letters

\bibitem{adam}
Diederik P. Kingma,
Jimmy Lei Ba. 
\textit{Adam: A method for stochastic Optimization} 
arXiv:1412.6980v9 [cs.LG] 30 Jan 2017


\bibitem{SGcomparativa}
Sebastian Ruder
\textit{An overview of gradient descent optimization
algorithms}
arXiv:1609.04747v2 [cs.LG] 15 Jun 2017


\end{thebibliography}



% �ndice alfab�tico%
\input{F1.Index.tex}

\end{document}
