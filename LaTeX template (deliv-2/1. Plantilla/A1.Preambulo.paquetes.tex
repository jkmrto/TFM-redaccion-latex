%%%%%%%%%%%%%%%%%%%%%%%%%%%%%%%%%%%%%%%%%%%%%%%%%%%%%%%%%%%%%%%%%%%
%%% Documento LaTeX 																						%%%
%%%%%%%%%%%%%%%%%%%%%%%%%%%%%%%%%%%%%%%%%%%%%%%%%%%%%%%%%%%%%%%%%%%
% T�tulo:	Paquetes
% Autor:  Ignacio Moreno Doblas
% Fecha:  2014-02-01
%%%%%%%%%%%%%%%%%%%%%%%%%%%%%%%%%%%%%%%%%%%%%%%%%%%%%%%%%%%%%%%%%%%
% Tabla de materias:
%	1 Codificaci�n e idioma %
% 2 Matem�ticas y F�sica %
% 3 Gr�ficos%
% 4 Estilo y formato%
%%%%%%%%%%%%%%%%%%%%%%%%%%%%%%%%%%%%%%%%%%%%%%%%%%%%%%%%%%%%%%%%%%%

%1 Codificaci�n e idioma%
\usepackage[latin1]{inputenc} %Codificaci�n en latin-1%
\usepackage[spanish]{babel}	%Hyphenation (Guionado) en espa�ol%
\usepackage[T1]{fontenc} %Codificaci�n de fuente%
\usepackage{eurofont} %Tipograf�a euro (�)%

%2 Matem�ticas y F�sica %
% Importante para ecuaciones, magnitudes y unidades%
\usepackage{amssymb,amsmath,latexsym,amsfonts} % paquetes est�ndar%
\usepackage[squaren]{SIunits} %Paquete para magnitudes y unidades f�sicas%
\usepackage{ifthen} %sentencias if y while%

%3 Gr�ficos%
\usepackage{graphics,graphicx} %paquetes gr�ficos est�ndar%
\usepackage{wrapfig} %paquete para gr�fica lateral%
\usepackage[rflt]{floatflt} %figuras flotantes%
	% \begin{floatingfigure}[r]/[l]{4.5cm}
	% \end{floatingfigure}
\usepackage{graphpap}	%comando \graphpaper en el entorno picture%

%4 Estilo y formato%
\usepackage{fancyhdr}	%cabeceras y pies mejor que con \pagestyle{}%
\usepackage{titlesec,titletoc} %Formateo de secciones y t�tulos%
\raggedbottom %Para fragmentar versos en varias p�ginas%
\usepackage{makeidx} %MakeIndex%
%\usepackage{showidx} % Hace que cada comando \index se imprima en la p�gina donde se ha puesto (�til para corregir los �ndices)
\usepackage{alltt} % Define el environment alltt, como verbatim, excepto que \, { y } tienen su significado normal. Se describe en el fichero alltt.dtx.
\usepackage[pdftex,bookmarksnumbered,hidelinks]{hyperref} %hyper-references%
\usepackage{minitoc} % Para poner tablas de contenido en cada cap�tulo.
\usepackage{listings} % Para escribir piezas de c�digo C, Python, etc. %
%listings configuration
\lstset{
  language=Python, %Puede ser C, C++, Java, etc.
  showstringspaces=false,
  formfeed=\newpage,
  tabsize=4,
  commentstyle=\itshape,
  basicstyle=\ttfamily,
  morekeywords={models, lambda, forms}
}

\usepackage{tipa} % tipograf�a IPA (International Phonetic Alphabet)
\usepackage{longtable} %Entorno Longtable, fracciona tablas a lo largo de p�ginas%
\usepackage{colortbl}
\usepackage{acronym}  %Para expandir autom�ticamente los acr�nimos