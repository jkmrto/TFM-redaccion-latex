%%%%%%%%%%%%%%%%%%%%%%%%%%%%%%%%%%%%%%%%%%%%%%%%%%%%%%%%%%%%%%%%%%%
%%% Documento LaTeX 																						%%%
%%%%%%%%%%%%%%%%%%%%%%%%%%%%%%%%%%%%%%%%%%%%%%%%%%%%%%%%%%%%%%%%%%%
% T�tulo:		Cap�tulo 2
% Autor:  	Ignacio Moreno Doblas
% Fecha:  	2014-02-01
% Versi�n:	0.5.0
%%%%%%%%%%%%%%%%%%%%%%%%%%%%%%%%%%%%%%%%%%%%%%%%%%%%%%%%%%%%%%%%%%%
\chapterbegin{Fundamentos Te�ricos}
\label{chp:Utiliz}
%\minitoc

\par El trabajo realizado en este proyecto es englobado dentro de la tem�tica denomiada visi�n por computador, dado que los m�todos empleados se basan en la detecci�n de patr�nes sobre las im�genes dadas, en nuestro caso neuroim�genes. 

\par En el �mbito de la visi�n por computador cada imagen en s� misma es una muestra de miles dimensiones, cada uno de los pixeles. Un modelo generativo trata de capturar la relaci�n entre las multiples dimensiones de los datos. En nuestro caso el modelo empleado para capturar dichas relaciones el Autoencoder Variacional.

\par Es por ello que este cap�tulo se centrar� en la exposici�n de este m�todo en primer lugar y posteriormen se explicar�, de forma breve, otros m�todos usados de manera auxiloiar en el desarrollo del proyecto. 

\section{Autoencoder Variacional}

\subsection{Modelo de Variables Latentes}
\par A lo largo del entrenamiento o la caracterizaci�n de un modelo generativo, la parte mas complicada es la extracci�n de las dependencias entre las m�ltiples dimensiones. Son estas relaciones multidimensionales las que permiten generar muestras artificiales pertenecientes a clases distintas. Se denomina variable latente, a las unidades del modelo generativo capaces de discernir entre las distintas clases, esto es, capacitan al modelo para generar elementos diferenciados.
\par Un modelo generativo es representativo de un espacio muestral $(X)$ si para cada una de las muestras de dicho espacio $(x)$ hay al menos alguna configuraci�n de las variables latentes $(z)$ que genera un variable $(\hat{x})$ muy similar a la original. Formalmente, dada una funci�n  $f(z, \theta)$ parametrizada por un vector $\theta$ en un espacio $\Theta$ tal que:
\begin{center}
\begin{equation} \label{eq:space}
f : 	Z \times \Theta  \rightarrow  X  
\end{equation}
\end{center}
\par El objetivo es maximizar la probabilidad de cada $x$ de el espacio muestral de acuerdo con:
\begin{center}
\begin{equation} \label{eq:int_1}
P(X)  = \int_{}^{}P(X|z;\theta) 
\end{equation}
\end{center}
\par En la ecuaci�n \ref{eq:int_1}, $f(z;\theta)$ es reemplazada por la distribuci�n $P(X| z;\theta)$, la cual nos permite hacer expl�cita la dependencia de $X$ sobre $z$, debido a la probabilidad condicionada. 
La idea de detr�s de dicha expresi�n es principio de m�xima verosimilitud (ML, del ingl�s \textit{Maximum Likehood}), el cual indica que si el modelo es capaz de generar muestras del espacio $X$, entonces ser� probable que le modelo generativo construya muestras similares.

\par En el VAE, la funci�n de probabilidad $P(X|z;\theta)$ es las siguiente:

\begin{center}
\begin{equation} \label{eq:p_x_gausiana}
P(X|z; \theta)  = N(X| f(z;\theta), \sigma^{2}*I) 
\end{equation}
\end{center}

\par El uso de una distribuci�n gausiana nos permite realizar un descenso en gradiente durante la optimizaci�n, con objeto de caracterizar el modelo. Esta caracterizaci�n permite incrementar $P(X)$, entendidada como la probabilidad global de generar alg�n tipo de muestra de dicho espacio. Esto no ser�a posible si esta funci�n de probabilidad fuera una delta de Dirac. 






\newpage
\section{Herramientas Complementarias}


\chapterend{}